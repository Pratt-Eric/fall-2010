\documentclass{article}

\usepackage{amsmath,amsthm,amssymb}
\usepackage{mathrsfs}
\usepackage{geometry}
\usepackage{verbatim}
\usepackage{listings}
\usepackage{color}
\usepackage{textcomp}
\usepackage{setspace}
\usepackage{palatino}
\usepackage{hyperref}
\usepackage{pdfpages}

\geometry{top=1.3in, bottom=1.3in, left=1.2in, right=1.2in}

\renewcommand{\lstlistlistingname}{Code Listings}
\renewcommand{\lstlistingname}{Code Listing}
\definecolor{gray}{gray}{0.5}
\definecolor{green}{rgb}{0,0.5,0}
\lstset{
language=python,
basicstyle=\ttfamily\scriptsize\setstretch{1},
stringstyle=\color{red},
showstringspaces=false,
alsoletter={1234567890},
otherkeywords={\ , \}, \{},
keywordstyle=\color{blue},
emph={access,and,break,class,continue,def,del,elif ,else,%
except,exec,finally,for,from,global,if,import,in,i s,%
lambda,not,or,pass,print,raise,return,try,while},
emphstyle=\color{black}\bfseries,
emph={[2]True, False, None, self},
emphstyle=[2]\color{green},
emph={[3]from, import, as},
emphstyle=[3]\color{blue},
upquote=true,
morecomment=[s]{"""}{"""},
commentstyle=\color{gray}\slshape,
emph={[4]1, 2, 3, 4, 5, 6, 7, 8, 9, 0},
emphstyle=[4]\color{blue},
emph={[5]sys, numpy, os, cvxmod, cvxopt, random},
emphstyle=[5]\color{green},
literate=*{:}{{\textcolor{blue}:}}{1}%
{=}{{\textcolor{blue}=}}{1}%
{-}{{\textcolor{blue}-}}{1}%
{+}{{\textcolor{blue}+}}{1}%
{*}{{\textcolor{blue}*}}{1}%
{!}{{\textcolor{blue}!}}{1}%
{(}{{\textcolor{blue}(}}{1}%
{)}{{\textcolor{blue})}}{1}%
{[}{{\textcolor{blue}[}}{1}%
{]}{{\textcolor{blue}]}}{1}%
{<}{{\textcolor{blue}<}}{1}%
{>}{{\textcolor{blue}>}}{1},%
framexleftmargin=1mm, framextopmargin=1mm, frame=shadowbox, rulesepcolor=\color{gray}}
% Configuring hyperref package; 
\hypersetup{
    bookmarks=true,         % show bookmarks bar?
    unicode=false,          % non-Latin characters in Acrobat’s bookmarks
    pdftoolbar=true,        % show Acrobat’s toolbar?
    pdfmenubar=true,        % show Acrobat’s menu?
    pdffitwindow=false,     % window fit to page when opened
    pdfstartview={FitH},    % fits the width of the page to the window
    pdftitle={My title},    % title
    pdfauthor={Author},     % author
    pdfsubject={Subject},   % subject of the document
    pdfcreator={Creator},   % creator of the document
    pdfproducer={Producer}, % producer of the document
    pdfkeywords={keywords}, % list of keywords
    pdfnewwindow=true,      % links in new window
    colorlinks=false,       % false: boxed links; true: colored links
    linkcolor=gray,          % color of internal links
    citecolor=green,        % color of links to bibliography
    filecolor=magenta,      % color of file links
    urlcolor=cyan           % color of external links
    linktoc=all             % link in table of contents
}
% table of contents setups. 
\makeatletter
\renewcommand\l@section[2]{%
  \ifnum \c@tocdepth >\z@
    \addpenalty\@secpenalty
    \addvspace{1.0em \@plus\p@}%
    \setlength\@tempdima{1.5em}%
    \begingroup
      \parindent \z@ \rightskip \@pnumwidth
      \parfillskip -\@pnumwidth
      \leavevmode \bfseries
      \advance\leftskip\@tempdima
      \hskip -\leftskip
      #1\nobreak\ 
      \leaders\hbox{$\m@th\mkern \@dotsep mu\hbox{.}\mkern \@dotsep mu$}
     \hfil \nobreak\hb@xt@\@pnumwidth{\hss #2}\par
    \endgroup
  \fi}
\makeatother

\begin{document}
\tableofcontents
\newpage

\section{Objective of Experiment}
This experiment is used to test the performance privacy preserving
outsouring algorithm. \\
We compare the query time of original data as compared with
transformed data with varying data size and vary dimensions. \\
For purpose of efficiency, we utilized the RTree structure to store
and spatial data. 

\section{Experiment Design}
\subsection{Experiments}
\lstinputlisting{experiments}

\subsection{Workflow of Experiment}
Work flow control script:
\lstinputlisting[language=bash]{doit.sh}
%\includepdf{flowchart.pdf}      % TO BE ADDED

\subsection{Data Structures}
\lstinputlisting{README}

\section{Random Parameter Generator}
In order to generate random queries and transform data and queries, we
need parameters $A$ and $v$.  
A is a random matrix with dimension $(k+2)*(k+2)$, and A is
inversable. And $v\in(0,1]$. \\ 
The code is listed below: \\
\lstinputlisting{generatp.py}

\section{Random Query Generator}
\lstinputlisting[language=python]{generatq.py}

Format of the random query matrix:\\
\begin{equation}
\begin{pmatrix}
$$(S_{11},E_{11}) & (S_{12},E_{12}) & \dots & (S_{1k},E_{1k}) \\
(S_{21},E_{21}) & (S_{22},E_{22}) & \dots & (S_{2k},E_{2k}) \\
\hdotsfor[2]{4} \\
(S_{n1},E_{n1}) & (S_{n2},E_{n2}) & \dots & (S_{nk},E_{nk})$$
\end{pmatrix}
\end{equation}
Here, $S_{ij}$ is the start of the query element and $E_{ij}$ is the 
end of the query element, with $i\in[1,m]$ and $j\in[1,n]$.\\
In the end, I need to write the generated random query matrix to a
file for further usage. \\
For ease of generation and use, I used another format to generate this
random matrix. I use matrix S and matrix E to store the start and end
elements respectively. 
Below is the code: 
\lstinputlisting{generatq.py}
\section{Query Convertor}
\lstinputlisting{transq.py}

\section{Data Convertor}
\lstinputlisting{transd.py}

%\section{Data Analyzer}
\section{Linear Scan Verification}
\lstinputlisting{linearVerify.py}

\section{Data Filtering}
In order to remove data that fall into the result set because of
error, we need to do some filtering to the result according to the
following rule: 
\begin{equation}
y^{T}[(A^{-1})^Twv^TA^{-1}]y <= 0
\label{eq:filter}
\end{equation}
In this step, three files are needed: 
\begin{enumerate}
\item The parameter file which provides A; 
\item The original query file to construct w; 
\item The transformed data file, which contains the queried results; 
\end{enumerate}
For each line of the original query file, we do the calculation of
\ref{eq:filter}, if it complies with rule, then passed the filtering
and output the item id into the result file.\\

Code for query result filtering:
\lstinputlisting{filtering.py}
\subsection{Dilemma of filtering and an alternative}
While the filtering.py seems reasonable and works well, we need to
compare filtering process with the RTree querying process which is
implemented in C/C++, this makes it urgent for us to migrate the
matrix code within python to the one in C/C++. I have been trying
to use matrix libraries into my code. 
There are different kind of matrix libraries out there in C/C++. And
the performance of these libraries varies significantly. Some of them
are pretty simple to use, while others use the complex C++ class
hierarchy which makes it not a simple task to utilize within just a
few days. I have tryied to use newmat10 library, but scared away but
the complexity. But luckily, I decided to program my own matrix
multiplication library. Actually, my code is just a specific
implementation of the matrix multiplication problem. 
Below are the steps that I implemented this: 
Steps of evaluation: 
\begin{enumerate}
1: (A^(-1))^T * w = a (kx1)
2: v^T * (A^(-1)) = b (1xk)
3: a * b = c (kxk)
4: y^T * c = d (1xk)
5: result = d * y (1x1)

\section{Conclusion}
This section is a conclusion of this experiment. 

\section{Best Practices}
\begin{enumerate}
\item Make sure to use the \textbf{file.close()} method if the same file are
  accessed more than once, and an alternative way is to close the file
  and open it again, but this method will be a little more costly. 
\item For purpose of verification and debuging, try to break the
  whole system into modules, and verify these modules one by one and
  combine them together using scripting languages, such as bash etc.,
  and automate the experiment process. 
\item Data should be properly stored, such as into files through
  redirection or directly write it into files with bash scripts. 
\item There are differences between c++ and python as far as the
  running speed is concerned. So, it is reasonable to use the same
  code if the running efficiency are considered experiment parameter.
\end{enumerate}

\newpage
\lstlistoflistings

\end{document}