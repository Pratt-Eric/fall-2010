\documentclass[12pt]{article}
% Change "article" to "report" to get rid of page number on title page
\usepackage{amsmath,amsfonts,amsthm,amssymb}
\usepackage{setspace}
\usepackage{Tabbing}
\usepackage{fancyhdr}
\usepackage{lastpage}
\usepackage{extramarks}
\usepackage{chngpage}
\usepackage{soul,color}
\usepackage{titlesec}
\usepackage{engord}
\usepackage{graphicx,float,wrapfig}
\usepackage{url}

% In case you need to adjust margins:
\topmargin=-0.45in      %
\evensidemargin=-1in     %
\oddsidemargin=0in      %
\textwidth=7in        %
\textheight=9.0in       %
\headsep=0.25in         %

% Start of document. 
\begin{document}
\begin{center}                  % Header of file. 
\textbf{\Large{Privacy Protection of Social Networks - A Survey}} \\ 
\small\textsc{Shumin Guo} \\
\today
\end{center}

\begin{abstract} 
% background of WBSN privacy. 
The development of web based social networks(WBSN) are attracting more and more
users and are now among the most popular sites on the
web\cite{top-500-sites}.  These large WBSNs provide a
basis for maintaining social relationships, for finding users with
similar interests and for locating content and knowledge that has been
contributed or endorsed by other users\cite{measure-analyze-SN}.
And these sites also provides business opportunities for a 
variety of business entities, revolutionizing traditional business
model. While on the other hand, privacy has become a big concern for
these WBSNs. 

% What is this paper about? 
This paper is an overview of current research on the privacy of social
networks. We analyzed various factor that can cause privacy breach
and discussed current privacy research attempts for WBSN
privacy protection. We also discussed the difficulties for social
network privacy research. And some interesting privacy related
research problems are listed. 
\end{abstract} 

%% Introduction part. 
\section{Introduction}
% Social networks are becoming more and more popular. 
With the advent of web 2.0, social networks are becoming more and more
popular, it provides people with a virtual world to keep in touch with
friends, colleagues and relatives etc. And it is attracting more and
more users, changing the way people communicate, work and
entertain. Facebook \cite{poke-facebook}, for example, has attracted more
than 500 million users \cite{facebooksite} from its publish in
2004. And as it grows, users both from the US and all around the
world are joining this online community. Still, there are a lot of
other social networking sites, such as MySpace \cite{myspacesite},
which concentrates on music and entertainment, LinkedIn which focuses
on working professionals and Twitter \cite{twittersite}, a networking
service that let members post short, 140-character messages called
tweets. These social networking sites gains great popularity in this
social network era.

% The potential privacy issues and attracts more concerns. 
It is common that social network sites always try to entice users to
provide as much personal information as possible\cite{linkedin} in
order to utilize the services provided by social networks. Once
disclosed, this sensitive information can be abused by
adversories. So, similar to general data privacy issues there is a
balance between utility and privacy protection. They may use this
information to make context aware spamming\cite{context-aware-spam}
attacks with a very high success rate \cite{social-phishing}. Also,
users' activities on the social work such as posting personal
messages, making comments on friends' posts, photo uploading etc., if
not properly protected and controlled for controlled access, can also
lead to violation of personal privacy. Several years ago, Facebook's
Beacon, which correlate users' shopping activities on other websites
with their Facebook profiles, was forced to be turned off because of
users' protesting on personal privacy violation
\cite{facebook-turnoff-beacon}. And several other privacy incidents
were also reported over facebook
\cite{social-networking-report-economist}. Recently, Buzz, google's
social networking service within gmail, was even faced with lawsuit
because of personal profile leakage\cite{google-buzz-lawsuit}.

% Social platform and pervasive computing. 
Social networking platforms integrate third-party content into social
networking sites and give third-party developers access to user
data. So it has the potential of becoming another privacy killer of
social networks \cite{privacy-social-platform}. Although, most of the
leading social platforms have primitives for privacy protection, they
are not sufficient, and several literature papers have been trying to
deal with privacy protection for social platforms
\cite{privacy-fundamental-construct} \cite{xbook-social-platform}
\cite{privacy-social-platform} \cite{secure-social-app-framework}
\cite{user-privacy-social-app}. Social networking on pervasive computing
devices such as mobile phones, brings about the specific privacy of
users' location, \cite{privacy-mobile-SN} is trying to design a
framework to deal with this issue.  

% contradiction between users' expectations and social network
% providers makes privacy protection a issue. 
Over recent years, all kinds of privacy incidents and past research
work tell us that users have stronger and stronger expectations
for privacy on social network sites. And they believe that social
network sites have the responsibility to protect their
data\cite{privacy-social-platform}. But the incentives for social
network providers usually make them reluctant to align with users'
expectations, and thus untrusted from user's point of view. A
privacy-enabling social network framework over untrusted networks is
proposed in \cite{privacy-SN-untrusted-networks}. The goal is to
protect private information both from social network providers and
other users within the social network. 

% Challenges. 
\textbf{Challenges of user privacy protection on social networks}

Unlike traditional privacy and security issues, privacy on social
networks has more challenges. Firstly, social network is an open
community, so it is open to everybody, which brings advantages to
malicious users over traditional systems, where centralized control is
possible. Secondly, social network users are ad-hoc entities, so they
are assumed to have the sense of privacy protection on this large
network, while research shows that a lot of users don't even have this
sense of privacy protection\cite{privacy-wizard}, this on the one hand
is due to users' lack of privacy leakage consequences and on the other
hand, social network users provide insufficient control choices to
users and finally, average social network users don't have enough
knowledge about the access control configurations and implications
that these configurations might incur\cite{privacy-wizard}
\cite{Anwar_visualizingprivacy}. Thirdly, social network is an
interaction network, the social as well as interaction
graph\cite{user-interaction-social-link} have the potential of being
traversed, which means that we not only need to consider the user
whose privacy is to be protected, but also consider random user search
and traversal through the friendship network graph. And thus a group
based collective privacy protection scheme should be used. And
finally, social network is an open platform, new applications and new
features are emerging everyday, e.g. \cite{social-healthcare-privacy}
discussed how to protect user's privacy and security of health
information. Also, there should be a generalized privacy evaluation
and quantification standard so that people can have the ability to
know the risks of their private data. Tran Hong Ngoc et
al. \cite{SN-quantify-privacy}, Maximilien, E. Michael
\cite{privacy-fundamental-construct} and Liu, Kun et
al. \cite{SN-privacy-score} have done research on this issue.

% Structure of this paper. 
The rest of this article is organized as follows. Section
\ref{sec:threats} introduces privacy threats of online social
networks. Section \ref{sec:crypto} discusses cryptographic methods for
privacy preserving social networks. Section \ref{sec:accon}
introduces access control based based methods to control privacy of
social networks. Secion \ref{sec:platform} disscusses specific
privacy issues when social networks are used as an application
platform. And Section \ref{sec:summary} will summarize this review and
point out possible research topics for online social networks.

% Identify various threats of online social networks. 
\section{Privacy Issues of Online Social Networks(OSN)\label{sec:threats}}
% OSN users are facing various social threats.  
OSN users share a large amount of personal information, communication
content and photoes etc., and they think that social network sites
have developed detailed sets of privacy controls that they can
trust. But users' privacy is usually undermined either because of the
defects of privacy control technique of OSNs or because of the
ignorance of privacy protection. Still, third party malicious users
can user various social attacks to get users' private information.

Generally, three principal factor are responsible for the disclosure of
private information of OSNs\cite{threats-to-privacy}: users disclosure
too much information, OSNs don't private sufficient measure to protect
user privacy and the active seeking of private information by
malicious third parties. So, in this section, we will discuss these 
factors and find out reasons for privacy breach. 

\subsection{Too Much Disclosure of Private
  Information \label{subsec:infodisclosure}}
% obsecure settings turn users away on privacy settings. 
Although research and user surveys \cite{social-phishing}
\cite{granular-privacy-control} \cite{imagined-facebook-privacy} point
out that OSNs users have strong desire to control their private
information, the expressiveness and usability of privacy control
techniques provided by social network providers have scared users
away. Research conducted by \cite{imagined-facebook-privacy},
\cite{characterizing-privacy} and \cite{privacy-suites} have found
that over $80\%$ of social network users leave their privacy settings
as default and even less than $1\%$ of users change privacy setting to
avoid obsecure privacy violation features on
facebook\cite{prying-data-from-SN}. 

% What it means for default privacy settings. 
The default privacy settings of OSNs can bring large privacy
leakage possibilities. Usually, the default setting, like facebook, are
very relaxed, which means that a lot of users can have access to their
information. And intuitively speaking, the more open your data is to
other people, the less privacy you have. \cite{characterizing-privacy}
studied social networks from the point of view of data exposure, and
found that much user data can be crawled during to default privacy
settings. Crawls of social network data are even published by
\cite{parallel-crawl-SN} and \cite{measure-analyze-SN}. So, the
disclosure of private information makes it easy for adversaries to get
such information and privacy is greatly compromised at the same time. 

% How to deal with this problem? 
In order to deal with this problem, methods need to be taken to arouse
people's awareness of private issue of OSNs. And more importantly,
more expressive and usable privacy control settings user interfaces
should be designed to ease the use of once obsecure privacy
settings. In section \ref{sec:accon} we will discuss several such
methods that has been done by researchers. 

%Social networks can be used on businesses. And a lot of small
%businesses become bigger. 

\subsection{Defects of OSNs that can Cause Private Information
  Breach \label{subsec:OSNDefects}}
% Defects of social networks. 
Facebook’s fundamental belief is that users should be able to control
what is visible to the public and who within their social graph can
see specific information \cite{granular-privacy-control}. But defects
of social network itself, including what we have discussed in section
\ref{subsec:infodisclosure}, are also responsible for possible privacy
leakage incidents of OSNs. In this subsection, we will discuss these
defects and point out possibilities that it can cause privacy breach.

% most important problem is the business model. 
Business model \cite{business-model} of social networks is about
monetizing social networks. The goal of marketing strategies over OSNs
is to achieve revenue maximization through maximized social influence
\cite{SN-marketing}. A large number of research papers
\cite{max-revenue-single-sample} \cite{SN-marketing-candidate} have
discussed how to achieve this goal through social influence. This
business model makes OSN providers reluctant to enforce strict privacy
control policies, and sometimes they even achieve their goal by
sacrificing users' privacy. For example, in 2007 facebook breached
users privacy information by tracking users purchase activaties and
send alert messages to their friends without acknowledgement from user
\cite{social-networking-report-economist}. Sometimes, OSN providers
even sell users' information data to third parties for unpredicted
use. The goal of this business model makes other privacy protection
methods inefficient or even makes them no use. So, a new business
model or social networking architecture need to be designed so that
organization that host users' private data is neutral to the data it
is hosting. The emergence of cloud computing\cite{cloud-computing} and
development of distributed computing architecture made this idea
possible. Jonathan Anderson et al propose an architecture to protect
user's private information from both OSN operators and other OSN
users\cite{privacy-SN-untrusted-networks} and Kapil Singh et al. is
trying to use information flow model to control
privacy\cite{xbook-social-platform}. 

% Issues when OSNs are used as app platforms by third-parties. 
Besides of serving as communication platform, OSN sites,
such as facebook and myspace\cite{myspacesite}, are platforms for
various social applications such as online social games
\cite{xbook-social-platform} \cite{facebook-social-app}
\cite{privacy-fundamental-construct}. The rapid 
growth of these social applications will have the influence of how
content is produced and consumed over the social network, and thus
have serious privacy implications. To use social applications, OSN
users are asked to trust these third-party applications and authorize
them to access their private data. This has left the users' private
information vulnerable to accidental or malicious leaks by these
applications. 

% summarize. 
OSN providers and social platform can be major factor for private
breach, in section \ref{sec:platform} we will discuss current attempts
to deal with this issues and make our own proposals. 

% Third-party, usually malicious users. 
\subsection{Privacy Breach By Third Parties\label{subsec:thirdparty}}
Third parties can pose serious threats to the privacy of OSN
users. They can use various attacks\cite{SN-seven-deadly-attacks},
such as spamming and phishing attacks, infrastructure attacks, malware
attacks, identity theft attack, neighbor attacks and other social
attack methods to get users' private information. In this subsection,
we will discuss spamming, phishing, identity theft and neighbor
attacks in detail.

% spamming and phishing. 
\textbf{Spamming and Phishing\cite{spamming-phishing-privacy}} 

% Spamming 
Spammers and content polluters are threatening privacy of OSN
users. Studies on the features of social spam and measures to
effectively filter out these spams has been studied by Kyumin Lee et 
al. \cite{social-spammer-machine-learning}, Stringhini et
al. \cite{SN-detect-spam}, Gao, Hongyu et
al. \cite{SN-spam-campaigns} and Huber, Markus et
al. \cite{SN-explore-spam}. They find that the 
identified spam data contains contents that are strongly correlated
with observable profile features. This finding tells us that spammers
are trying to improve their spamming success rate by gleaning user
profile information using various ways, and then do context based
spamming. Once successful, users' OSN account can be compromised and
more sensitive or even private information might be stolen. 

% Benevenuto, Fabricio et al.\cite{video-spam-youtube} studied how to
% effectively identify video spams over youtube\cite{youtube}. 
% Grier, Chris et al. \cite{twitter-spam} studied the characteristics
% of spamming on twitter and find that twitter is a highly successful
% platform for spammers. They also propose if blacklisting can be used
% to filter out spams, but the result is not very promising. 

% Phishing. 
Phishing is a form of social engineering in which attackers attempts
to fraudulently acquire sensitive information by impersonating a
trustworthy third party. Phishing messages with malicious URL links or
attachments can be send to social network users. When users clicked
the malicious URL or download/execute the attachment, malicious
programs or websites, they can steal users' personal information or do
harm to users' computer. Tom Jagatic et al. \cite{social-phishing}
studied phishing attacks by using the publicly available personal
information from social networks. They find that the phishing attack
was easy and effective with a success rate of $72\%$.

% Counter measure against these attacks. ???
Spamming and phishing of social networks can pose high threat on
users' privacy, especially, when these techniques are fully automated 
nowadays\cite{SN-automated-cheap-spam}\cite{SN-explore-spam}. How to
effective and efficiently detect and filter these spam and phish
content from OSN still needs to be figured out.

% Social attacks. 
In addition to spamming and phishing, malicious third parties can use
various other attack methods to get users' private information. 

% identity threft attacks. 
\textbf{Identity Threft Attacks}

% literature by Bilge. 
Bilge, Leyla et al.\cite{identity-theft-attack} studied identity
theft attacks. They proposed that existing users of OSN can be
compromised, and their identity can be used to request friendship with
other cloned victims. And they prove that this attack can be done on
five most popular social networking sites by means of automated
profile cloning and cross-site profile cloning. And further experiments
results on real social network users confirm that this type of attack
can achieve a high success rate. 

% methods to avoid this kind of attack. 
Methods to avoid this kind of attack is many fold. We can use
authentication methods to authenticate users that are initiating
friendship requests. And social network service providers 
can adopt behavior based anomaly detection techniques and block
automatic crawling and other suspicious activities. But more research
work needs to be done on the details of these techniques. 

% In the following paragraphs we will discuss various social attacks. 
% De-anonymization attack. 
\textbf{De-Anonymization Attacks}

% anonymization can cause de-anonymization attacks. 
Social network providers hosts large amount of sensitive data from
OSNs users. The sensitive data providers a good source both for the
research and for the business, so there is a require that data be
analyzed by social network providers itself or unknown third
parties. Past work on privacy preserving data publishing
\cite{ppdp-survey} have proposed several methods to deal with this
issue. But for concern of simplicity, current data publishing of
social network data usually use the anonymization method to shield
sensitive information. But research by Lars Backstrom 
et al. \cite{anony-link-attack} and Gilbert Wondracek et al.
\cite{group-deanonymization-attack} who that anonymization methods can
cause breach of user privacy. 

% De-anonymizing social links. 
Lars Backstrom et al. \cite{anony-link-attack} show that adversaries
can use anonymized social data to infer links between social network
users. Both passive and active attack are shown to be effective to do
this de-anonymization attack. With prior knowledge of the the social
network and the de-anonymized links, attackers can identify a specific
user. 

% threats of link deanoymization. 
Also, the de-anonymization of social links brings privacy threats to
social network users.  \cite{user-interaction-social-link} proposes
the use of interaction graph to impart meaning of social links by
quantifying user interactions, and showed that interaction graph can
be used as a better representation of user interactions of social
network users. Similarly, \cite{neighborhood-attack} dicusses
neighborhood attacks. The identity of a specific user can be
identified by obtaining some knowledge about the neighbors of a
targeted victim and the relationship among these neighbors.

Lian Liu et al.\cite{privacy-sensitive-edge} have studied privacy
issue of sensitive edge weights of social networks and proposes
methods based on randomization and purtabation for privacy
preservation.

% Group information based user de-anonymization. 
Gilbert Wondracek et al.\cite{group-deanonymization-attack} proposed an
attack model which uses group information to De-Anonymize users from
social network. By obtaining the group membership information of a
particular user, they show that it is usually sufficient to uniquely
identify the identity of this user. In the group information based
de-anonymization attack, web browsing history information is utilized
to obtain group information of a certain user. Experiments show that
little effort is needed to do De-Anonymization attacks, so it has the
potential to affect millions of social network users with group
memberships.

% research about it. 
De-anonymization attacks of social network data and methods to
aliviate this attack is still an active research topic. One possible
method is to try to normalize social network data and use the
relatively mature PPDP methods. 

\textbf{Sybil Attack} 

In large scale peer-to-peer systems, redundancy can be used to combat
scurity threats from malious remote systems, but John et
al. \cite{sybil-attack} show that the identity of these systems can
be forged to undermine this redundancy, which is called Sybil
attack. And they also show that Sybil attacks can always be possible
for a system without a trustable centralized authority. 

Social network are composed of independent users forming a large
peer-to-peer system, and it is usually problematic \cite{sybilguard} to
implement a centralized authority in such a system. Even worse, it is a
rapidly growing dynamic system. So, these features make social
networks infectious to Sybil attacks. In social networks, malicious
users can obtain multiple fake identities and pretend to be multiple
nodes in the system. And by controlling the large number of user
identities, malicious users can "vote out" honest users in
collaborative tasks. Also attackers can solicite the sensitive
information usign fake identities thus causing the leakage of private 
information. Haifeng et. al\cite{sybilguard} proposed a decentralized
protocol that limits the corruptive influence of sybil attacks based
on human-established trust relation. This protocol bounds the number
and size the sybil attack groups by utilizing the trust relation among
social network users. Viswanath et al\cite{sybildefenses} uses
community structure for sybil attacks defense. It generalized the
Sybil attack schemes into graph partitioning algorithm, and by
defferent parameter settings, social network users are ranked
according to the trust relative to designated trusted nodes. 

\textbf{Auxilliary Data Attacks}

Often times, social network activity can be linked to real life
scenario. Minimum information is need for such kind of real life
linkage attack if a lot of personal information is known about the
victim. \cite{real-life-social-network} compares online social network
and real life social network and points out the domain feature of the
real life social network and influences of online social network to
the real life social network. This paper also proposes that different
types of relationships of real life social network is a good
implication of the design of online social networks for privacy
purposes. It also proposes that causual relationships of both online
and real life social network is common and further confirms that the
interaction among social network users, which is also proposed in
\cite{user-interaction-social-link} are better implication of privacy
concerns for online social network users. \\
\textbf{More content to be added.}

% Cryptographic privacy protection methods. 
\section{Cryptographic Privacy Protection \label{sec:crypto}}
Cryptogrpic means have been tried by various literature papers
\cite{noyb},\cite{reliable_email},\cite{group-deanonymization-attack}
for the purpose of privacy protection for privacy of social network
users. With this method, user profiles and communications among
friends are encrpted all the time. The access of private information
is controlled by key management methods. 

\textbf{Random Substitution Encryption}

Saikat et al.\cite{noyb} proposes an pseudorandom substitution method
to encrypt user's information. In this method, sensitive information
is splitted into groups, and then substituted by information from
other social network users. For example, assume social network user
John have sensitive tuple name, age, and sex (John, 25, M). This tuple
is first partitioned into two sub-tuples (John, 25) and (M), then
(John, 25) is substitute with (Michael, 39) and (M) is substituted
with (F) based on the encrypt indices. In this case, John can
distribute his key to his friends so that they can decrypt the
substitued information, while other users without key can't decrypt
these encrypted data. 

A dictionary is maintained for the mapping from a sequential index to
unique atoms and the corresponding frequency. The pseudorandom
sequence is generated by symmetric-key cipher. Data is sent via four
channels either out-of-band or in-band, and each channel either for
key distribution or for the transmission of residue information, or
transmission of the online services. On the friend side, they will use
the key from the cipher friend to decipher the atoms. 

\textbf{Social Relation Based Cryptographic Method}

Scott et al.\cite{reliable_email} introduce an authentication
protocol for the problem of spam mails filtering. In this protocol,
secret key signed authentication tokens are attached to the mail. When
the receiver receives the mail, it will try to decide whether to
accept the mail or not based on the relationship between the sender
and the receiver. 

If the sender is direct friend of the receiver, it will verify the
authentication token through sender's attestation server. If the
verification succeeds, receiver will accepts the mail. And if the
sender and receiver are not direct friends, the receiver will try to
determine if sender is friend of friend. This is done by asking the
sender to check if there are any friends in the friends' list of the
receiver. But rather than sending the list of friends, receiver utilizes a
private set matching protocol to do the attestation process, and this
protocol can guarantee that the sender does not learn any of the
recipient's direct friends. 

Authentication tokens can also be used to detect forged senders. This
is done by verifying if the token is unused, the token's signature
matches the one owned by the sender through the attestation server of
the sender and finally, the recipient will check that the sender,
recipient and message id match the value found in the message
itself. If any of these conditions fails, the receiver will drop the
message. 

Amin et al. \cite{lockr} observe that the tight coupling of two roles,
social information management and social application, of social
network lead to challenges for privacy protection. So, they propose a
method to decouple the social information from its
funtionality. And similar to the method proposed by Scott et
al.\cite{reliable_email}, social attestations are used for purpose of
relationship setup and access control. 

Similarly, Randy et al. \cite{persona} try to use cryptographic method
to define a Attribute Based Encryption (ABE) protocol. And users use
the encrytption key to manage the access of groups. Jinyuan et
al. \cite{revocation-management} propose method for revocation of
access authority from social network users in the group
management. Carminati et al. \cite{crypto-collaborative-ac} propose a
privacy preservation technique for access control during the
collaborative process within social groups. 

Although cyptogrphy provides an effective privacy protection
method, how to effectively use this method, especially for the problem
of key management within the large scale system. Consider the
situation when a user has a large number of friends, in this case the
overhead of authentication and attestation might be overwhelming and
will undermine the efficiency of the information distribution of
online social networks. 

%
% Privacy protection using Access Control methods.
%
\section{Privacy Protection using Access Control Methods \label{sec:accon}}
Social networks allow users to publish personal data When users sign
into the social network sites, they need to provide their personal
profile information. These sensitive information are stored on the
social network service providers' database and usually they have whole
control of these sensitive data. They can use these data for
analytical or business purposes. And the data can also be sold to
othere parties for similar or unpredicated applications. Various
privacy preserving data publishing methods have been proposed
\cite{ppdp-survey}. And from social network users' perspective, they
need to define access control policies that only enable legitimate
users to view their profile and posts, and prohibit adversaries from
doing so. So, a precise and efficient access control protocol becomes
the core of user privacy control. In this paper we will focus on the
privacy control from social network users' perspective and summarize
and evaluate different access control protols, and study the
challenges of privacy control over social networks.

Social network sites such as Facebook provides options to users to
personalize their private settings to be different levels from among
private, friend, friend's friends and totally public. While on the
other hand, user's understanding about the implication of these
settings and the ease of these settings are also important factors to
user's settings. In fact, a lot of users are not aware of the privacy
access configuration and just leave it as default
\cite{privacy-wizard}\cite{imagined-facebook-privacy}\cite{privacy-from-audience},
and thus undermined these privacy control settings. So, how to
effectively let users configure their privacy with minimum interaction
has become one of the research issue related to the privacy control
mechanisms.

% To be purified. 
\subsection{Challenges for Access Control Privacy Protection}
Social Networks are built upon relations between users, so the access
control and privacy protection models should be built based on the
relationships among users. A lot of research proposals expressed the
access control requirements in terms of relationship paths existing in
the network and their depth. Also, some models support a notion of
trust/reputation as a further parameter for access control decisions.

Besides, the enforcement of relationship-based access control poses
interesting issues regarding privacy protection. A further issue is
the architecture on the support of access control. Unlike traditional
access control models such as rule based access control, the
centralized access control are no longer a suitable or efficient model
for OSNs\cite{}. A decentralized privacy-aware access control
mechanism should be devised to enforce not only relationship-based
access control but also ensure the relationship privacy.

According to a relationship-based access control model, access control
policies are specified in terms of relationships existing in the
OSN. Additionally, the depth of the relationship path is an important
parameter for some access control decisions, since users are usually
more inclined to share their resources with users not much away from
them. Still a further important parameter is represented by trust
among social network nodes.

Moreover, in OSN, there exists relationship between users and
resources. So, a relationship-based access control model should
exploit not only the standard user to user relationships but should
also consider the various relationships and connection between users
and resources.

As we have discussed earlier, a centralized access control model is
not suitable for OSNs, a decentralized model should be used, where
each user is responsible for policy specification and enforcement. But
the decentralized model need to verify the existence of specific paths
within an OSN. This kind of task may be very difficult and time
consuming in a decentralized manner. So, a further essential
requirement of access control enforcement is to devise efficient and
scalable implementation strategies.

On the other hand, the relationship-based access control also poses
threats to the relationships which are in general sensitive
information. As the relationship-based access control may require
disclosure of personal relationships. So, another requirement for
relationship-based access control model is to ensure that relationship
privacy is not breached during access control. Also, relationships may
have an associated trust value that must be protected during access
control. So, there will be a requirement to protect the relationship
trust level.

Also, the decentralized enforcement of access control should have some
mechanism to help a user to precisely estimate the other users's trust
level. And such mechanism should also preserve user privacy when
performing trust computation. 

\subsection{Access Control Methods for Online Social Networks}
% machine learning technique for access control. 
\textbf{Automatic Access Control with Active Machine Learning.}

Lujun et al. \cite{privacy-wizard} try to use machine learning
techniques to build a privacy preference model (privacy wizard) to
describe the preferences of users and then use this model to
automatically configure the privacy settings for users. By providing
very simple user interactions, the privacy 
wizard use active machine learning by iterarively asks users to assign
privacy labels to specific friends. The model is designed such that
the quality of the privacy setting classifier improves with more input
from user and will adapt with the addition of new
friends. Visualization technique as proposed by Mohd et
al. \cite{Anwar_visualizingprivacy} can also be helpful for social
network users to understand the implication and meaning of all the
privacy settings.

\textbf{Relationship Protecting Access Control.}

Sometimes, relationship protection arises when social network users
want to keep private their relationship with other users. And as in fact,
relationship among users can lead to privacy 
breach\cite{private-relationship} \cite{SN-private-relation}. So, this
leads the question of how to protect relationship within the
enforcement of access control. Carminati et al.\cite{crypto-collaborative-ac}
propose a relationship protecting access control method based on the
collaboration of nodes in the network. By utilizing the collaboration
among social network users, a path is provided to the owner
proving that the resource requestor has sufficient relationships
required to access to the resource. 

\textbf{Trust and Relationship Level Based Access Control.}

Carminati et al. \cite{enforce-access-control} propose a
semidecentralized access control protocol. It takes the
depth and types of user relationships as well as the level of trust
into consideration. Different from the traditional access control
systems, the verification whether an access is authorized
is done by the information requesting user, who must provide evidence
to the information owner that he/she satisfies the requirements
experessed by the owner's access control policies. Specifically, the
information requestor must provide the information owner with a proof
showing the existence of the required relationship, and the
relationships have the required depth and trust level. A proof of the
direct or indirect relationship of a given type between user A and
user B is generated and certified by through a set of certificates
confirming the existence of the path of that type between them. The
proof of trust level provided by resource requestor can be problematic
when resource requestor intentionally hide malicious path which has
low level of trust between these two users. In order to deal with this
problem a central Certificate Server is needed. This server is used to
store all the relationship certificates. In this case, when the
requestor needs to prove to the resource owner the existence of a
given relationship, as well as the depth and trust, the certificate
server can be used to discover the set of certificate paths
corresponding to all the shortest paths referring to the required
relationship. Then, the resource requestor use this signed certificate
paths from the certificate server to generate the proof. And then send
to the resource owner to verify the validity of the proof. 

By introducing the Certificate Server, the certificate management can
be more efficiently done than by individual social network users. And it
is scalable to the size of the social network. So by seperating the
management of certificates and certificate path discovery between the
Certificate Server and resource requestors, a trade off between
efficiency and security requirements. 

% Collaborative control. 
\textbf{Collaborative Access Control.}

Sometimes, contents produced on the social network can have multiple
related owners. For example, a picture of Alice and Bob. When Alice
upload the picture to her social network site, and make privacy
preference settings, Bob may agree or disagree with this setting. How
to protect the privacy under this co-ownership scenario is a 
challenging problem, because in this case, several users are involved
in a collaborative access control process. Thomas et
al. \cite{unfriendly} try to formalize this conflict and show that lack
of collaborative access control of co-ownership resources can lead to
privacy breach. For example, a user's personal attribute can be
predicted based on the contents of the user's friend. Squicciarini et
al. \cite{collective_privacy} propose a method to deal with such 
problem. By associating a value $v_i(g)$ with user $i$'s privacy
settings $g$, the collaborative value is defined by summing up the values
of all the co-owners of this resource, which is defined as:
$\sum_{i=1}^nv_i(g)$. By maximizing the social value, an optimal
privacy setting can be achieved with the help of maximization likelihood
estimation method. 

Carminati et al. \cite{crypto-collaborative-ac} proposed a
collaboration method between the resource requestor and resource
owner. The resource requestor try to gain access to resources by
providing the resource owner proof of certified access rules to the
resource owner. 

\subsection{Other Access Control Based Privacy Protection}
\textbf{\underline{This part is to be enriched.}}

In current available literatures, various access control models have
been proposed, such as ACL model, multi-level access control,
challenge-response-based model etc.

Also, there are proposals on distance based access control, in which
requestors are divided into adjacent zones, and different zones have
different rules or policies for access control. 
A lot of literature also proposed cryptographic methods which are used
for relationship certificate exchange, attestation and so on. And there
are also a formalized access control model proposed in the literature,
in which the authers proposed several types of access control
policies, resources access control, search policy, tranversal policies
and communication policies. And recent research on semantic web also
provides opportunities for access control research.

But, it is important to know that most of the research related to
privacy in OSNs have focused on privacy-preserving techniques to mine
social network data, only a few provide solutions for privacy aware
access control. Among these privacy aware access control models, some
are policy based, while others consider trust protection. 

\section{Social Networks as Platform \label{sec:platform}}
Social networking sites can not only serve as a communication
platform, it can be used as a platform for third party
applications. For example, gaming via social network is expected to be
a billion-dollar business on year 2011 \cite{billion-dollar-social-gaming}. Reports
\cite{zynga} said that Zynga, one of the most famous social gaming
provider, CityVille game already have 3 million daily active users. A
third party can use the published social API to develop their own
applications. Sensitive information accessed by third party
applications and be out of control by information owners. So how to
protect privacy for this scenario are in urgent need with increasing
number of applications and users joining the social cloud
platform. Past work by Kapil et al. \cite{xbook-social-platform} and
Adrienne et al. \cite{privacy-social-platform} propose methods to deal
with this problem. 

% xbook - redesigning the social network platform. 
Kapil et al. propose a framework for building privacy preserving
social network applications that retains the functionality offered by
current social networks. An information flow model is used in this
framework such that information received by untrusted third party can
be controlled. And thus leaking of private personal infromation can be
prevented. This work identifies three potential ways of private
information leakage. (1) Social applications may share information
with any other party, such as advertisers, malicious users etc.. (2)
Application can pass information from one application user to another
user, which might break restrictions that only authorized users can
view user's profile. (3) The application can recreate the social graph
of all users by connecting common friends in this graph. In order to
prevent these three possible private information leakage, this
framework breaks applications into components. And by enforcing
mediation of all communication with the external entities and between
communications between two components, (1) and (2) can be
prevented. (3) can be prevented by making sure that a component can
only have an anonymized view of all its user's data. 

%  
\textbf{More contents to be added.}

\section{Anonymizing Social Network\label{sec:anony}}
Smriti et al.\cite{dynamic-social-network} proposes the question of
privacy protection when publishing dynamic social network data. An
anonymization method, is proposed. It makes use of link prediction
algorithms to model the evolution of the social network, and use the
predicted graph to perform group based anonymization. 

\section{Formalizing Privacy Protection for Social Networks.}

% Conclusions of this paper. 
\section{Conclusion \label{sec:summary}}
In this survey, we discussed privacy threats of online social network
users, pointing out possible ways that information can be leaked to
malicious users and possible social attacks based on these sensitive
information. And we classify current attempts for online social
network privacy protection to two types: cyptographic and access
control, pointing out that the former methods brings overload to the
social network users and providers. The access control based access
control methods can utilize the power of machine learning techniques
to automate the configurations of privacy preferences. And initial
attempts show that this is better method for online social network
privacy protection. While on the other hand, access control of social
network is unlike the traditional access control systems. It is more
automatic, distributed and need more precise tunning to the content. 

Research in online social network privacy related area is still in its
infancy, there are a lot of issues to be explored. One is the design
of a satisfactory solution to relationship privacy protection during
path discovery. Another research direction is related to policy
administration. Due to the large size of social network and relations,
it is important to devise techniques and tools that can help user
evaluate the risk of unauthorized flows of information that the
specification of a policy or itsupdate may cause.

With the development of online social networks, privacy protection is
becoming one major concern. While on the other hand, social network
providers are usually reluctant to enforce strict privacy protection
policies because of business concerns, which undermines privacy
protection greatly. So, it might be advisable that social network
platforms are revised to allow third party as sensitive data
hoster. And on the other hand legal procedures can be used for the
enforcement of privacy protection for online social network users. 

\bibliographystyle{unsrt}
\bibliography{survey}

\end{document}