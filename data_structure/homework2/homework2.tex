\documentclass{article}
% Change "article" to "report" to get rid of page number on title page
\usepackage{amsmath,amsfonts,amsthm,amssymb}
\usepackage{algorithmic,algorithm}
\usepackage{setspace}
\usepackage{Tabbing}
\usepackage{fancyhdr}
\usepackage{lastpage}
\usepackage{extramarks}
\usepackage{chngpage}
\usepackage{soul,color}
\usepackage{graphicx,float,wrapfig}
\usepackage{pstricks,pst-node,pst-tree}
\usepackage{pdftricks}

\usepackage{listings}
\usepackage{color}
\usepackage{textcomp}
\definecolor{listinggray}{gray}{0.9}
\definecolor{lbcolor}{rgb}{0.9,0.9,0.9}
\lstset{
	%backgroundcolor=\color{lbcolor},
	tabsize=4,
	rulecolor=,
	language=c++,
        basicstyle=\ttfamily\large\setstretch{1},
        upquote=true,
        aboveskip={\baselineskip},
        columns=fixed,
        showstringspaces=false,
        extendedchars=true,
        breaklines=true,
        prebreak = \raisebox{0ex}[0ex][0ex]{\ensuremath{\hookleftarrow}},
        %frame=single,
        showtabs=false,
        showspaces=false,
        showstringspaces=false,
        identifierstyle=\ttfamily,
        keywordstyle=\color[rgb]{0,0,1},
        commentstyle=\color[rgb]{0.133,0.545,0.133},
        stringstyle=\color[rgb]{0.627,0.126,0.941},
}
% In case you need to adjust margins:
\topmargin=-0.45in      %
\evensidemargin=0in     %
\oddsidemargin=0in      %
\textwidth=7.0in        %
\textheight=9.2in       %
\headsep=0.25in         %

% Homework Specific Information
\newcommand{\hmwkTitle}{Homework\ \#2}
\newcommand{\hmwkDueDate}{Oct.\ 12th,\ 2010}
\newcommand{\hmwkClass}{Data Structure}
\newcommand{\hmwkClassTime}{TR\ 4:10-5:25pm}
\newcommand{\hmwkClassInstructor}{Meilin\ Liu}
\newcommand{\hmwkAuthorName}{Shumin\ Guo}
\newcommand{\answer}{\textbf{\\\underline{ANSWER:}\\}}

% Setup the header and footer
\pagestyle{fancy}                                                       %
\lhead{\hmwkAuthorName}                                                 %
\chead{\hmwkClass\ - \hmwkTitle}  %
\rhead{Page\ \thepage\ of\ \pageref{LastPage}}                          %
\lfoot{\lastxmark}                                                      %
\cfoot{}                                                                %
\rfoot{}                          %
\renewcommand\headrulewidth{0.4pt}                                      %
\renewcommand\footrulewidth{0.2pt}                                     %

%%%%%%%%%%%%%%%%%%%%%%%%%%%%%%%%%%%%%%%%%%%%%%%%%%%%%%%%%%%%%
% Make title
\title{\textmd{\textbf{\hmwkClass:\
      \hmwkTitle}}\\\normalsize\small{Due\ Date:\
    \hmwkDueDate}\\} 
\date{\today}
\author{\textbf{\hmwkAuthorName}}
%%%%%%%%%%%%%%%%%%%%%%%%%%%%%%%%%%%%%%%%%%%%%%%%%%%%%%%%%%%%%

\begin{document}
\maketitle

\section*{}
1.(10 points) Describe in pseudo-code, a linear-time algorithm for
reversing a queue Q. To access the queue, you are only allowed to use
the basic functions of the queue ADT defined as follows (Hint: Using a
stack, the basic stack functions defined in the textbook and in the
class). 

\begin{lstlisting}
class Queue {
public:
    int size();
    bool isEmpty();
    Object& front()
    void enqueue(Object o);
    Object dequeue()
};
\end{lstlisting}
\answer

\section*{}
2. 2.(20 points) Let T be the binary tree as shown in the following
figure.\\

\begin{figure}[htb]
  \begin{center}
    \pstree[levelsep=10ex]{\Tcircle{A} }{
      \pstree{ \Tcircle{B}\tvput{} }{
        \pstree{\Tcircle{D}\tlput{}} {
          \Tcircle{G}
          \Tcircle{H}
        }
        \Tcircle{E}\trput{}
      }
      \pstree{\Tcircle{C}}{
        \Tcircle{F}
      }
    }
    \caption{Tree Struture for Question Two.}
  \end{center}
\end{figure}

\begin{enumerate}
\item[(a)]Give the output of the preorder tree traversal of T. 
\item[(b)]Give the output of the postorder tree traversal of T.
\item[(c)]Give the output of the inorder tree traversal of T.
\item[(d)]Give the output of the level order tree traversal of T.
\end{enumerate}
\answer

\section*{}
3. (10 points) Draw a (single) binary tree T, such that
\begin{enumerate}
\item[(a)]Each internal node of T stores a single character
\item[(b)]A preorder traversal of T yields A B C D F G E
\item[(c)]An inorder traversal of T yields C B F D G A E
\end{enumerate}
\answer

\section*{}
4. (10 points) Let T be the tree as shown in the following figure
\begin{figure}[htb]
  \begin{center}
    \pstree[levelsep=10ex]{\Tr{\psframebox[framearc=.5]{A}}}{
      \pstree{ \Tr{\psframebox[framearc=.5]{B}}\tvput{} }{
        \Tr{\psframebox[framearc=.5]{E}}
        \pstree{\Tr{\psframebox[framearc=.5]{F}}\tlput{}} {
          \Tr{\psframebox[framearc=.5]{I}}
          \Tr{\psframebox[framearc=.5]{J}}
        }
      }
      \pstree{\Tr{\psframebox[framearc=.5]{C}}}{
        \Tr{\psframebox[framearc=.5]{G}}
        \Tr{\psframebox[framearc=.5]{H}}
      }
      \Tr{\psframebox[framearc=.5]{D}}
    }
    \caption{Tree Struture for Question Four.}
  \end{center}
\end{figure}
\begin{enumerate}
\item[(a)]Give the output of the preorder tree traversal of T.
\item[(b)]Give the output of the postorder tree traversal of T.
\end{enumerate}
\answer

\section*{}
5. (20 points) Answer the following questions (This question is for
graduate students only, but undergraduate students are welcome to
finish this question to get bonus points). 
\begin{enumerate}
\item[(a)]What is the minimum number of external nodes for a binary tree with
height h? Justify your answer. 
\item[(b)]What is the maximum number of external nodes for a binary tree with
height h? Justify your answer.  
\item[(c)]Let T be a binary tree with height h and n nodes, Show that
$log(n+1)-1 \leq h \leq \frac{(n-1)}{2}$
\item[(d)]For which values of n and h can the above lower and upper bounds on
h be attained with equality? 
\end{enumerate}
\answer

\section*{}
6. (10points) For the set of ${ 2, 6, 10, 9, 4, 1, 13}$ of keys, draw
binary search trees of height 2, 3, 4, 5, and 6.
\answer   

\section*{}
7. (15 points) Consider the Binary Search Tree (BST) to the following.
\begin{figure}[htb]
  \begin{center}
    \pstree[levelsep=10ex]{\Tcircle{50} }{
      \pstree{ \Tcircle{35}}{
        \pstree{\Tcircle{23}} {
          \pstree{\Tcircle{12}} {
            \Tcircle{5}
            \Tcircle{15}
        }          
          \Tcircle{30}
        }
        \pstree{\Tcircle{42}} {
          \Tcircle{39}
          \Tcircle{48}
        }
      }
      \pstree{\Tcircle{58}}{
        \pstree{\Tcircle{52}} {
          \Tcircle{51}
          \Tcircle{55}
        }
        \pstree{\Tcircle{73}} {
          \Tcircle{65}
          \Tcircle{89}
        }
      }
    }
    \caption{Tree Struture for Question Seven.}
  \end{center}
\end{figure}
\begin{enumerate}
\item[(a)](5 points)  Draw a circle around each node that would be visited in
a search for the value 40. 
\item[(b)](10 points)  Show how the tree would appear if the value 35 were
deleted. Assuming duplicates are stored in the left subtree. 
\end{enumerate}
\answer

\section*{}
8.  (10points). Suppose that myBST is an empty Binary Search Tree. The
following operations are performed in the given order. Draw the
resulting tree.  Assuming duplicates are inserted in the left
subtree. 
myBST.insert(20) \\
myBST.insert(28) \\
myBST.insert(40) \\
myBST.insert(12) \\
myBST.insert(25) \\
myBST.insert(23) \\
myBST.insert(24) \\
myBST.insert(16) \\
myBST.delete(23) \\
myBST.insert(24) 
\answer

\section*{}
9. (20 Points). Suppose you are given the following array, which
represents a complete binary tree:
\begin{table}[h]
  \begin{center}
    \begin{tabular}{|c|c|c|c|c|c|c|c|c|c|c|c|c|c|c|c|c|c|} 
      \hline Index &0&1&2&3&4&5&6&7&8&9&10&11&12&13&14&15&16 \\
      \hline Key&12&25&72&49&64&58&52&92&8&19&43&27&98&16&37&33&6\\
      \hline
    \end{tabular}
    \caption{Array Representation of a Binary Tree.} 
  \end{center}
\end{table}
\begin{enumerate}
\item[(a)] (5 points) Draw the complete binary tree represented by this
array. 
\answer 
\item[(b)] (10 Points) Is this binary tree a binary heap? If not, build a heap
out of this array using the BottomUpHeap() we learned in the class. 
\answer
\item[(c)] (5 points) List the values from the heap as they would be printed
out by an inorder traversal of the heap, where the \textbf{visit()} function
prints the value of the current node. 
\answer
\end{enumerate}

%\end{spacing}
\end{document}

%%%%%%%%%%%%%%%%%%%%%%%%%%%%%%%%%%%%%%%%%%%%%%%%%%%%%%%%%%%%%
