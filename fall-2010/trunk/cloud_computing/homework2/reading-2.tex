\documentclass[24pt]{article}
% Change "article" to "report" to get rid of page number on title page
\usepackage{amsmath,amsfonts,amsthm,amssymb}
\usepackage{setspace}
\usepackage{Tabbing}
\usepackage{fancyhdr}
\usepackage{lastpage}
\usepackage{extramarks}
\usepackage{chngpage}
\usepackage{soul,color}
\usepackage{titlesec}
\usepackage{graphicx,float,wrapfig}

% In case you need to adjust margins:
\topmargin=-0.45in      %
\evensidemargin=0in     %
\oddsidemargin=0in      %
\textwidth=6.5in        %
\textheight=9.0in       %
\headsep=0.25in         %

% Homework Specific Information
\newcommand{\hmwkTitle}{Reading\ \#2}
\newcommand{\hmwkDueDate}{Monday,\ Oct.\ 01,\ 2010}
\newcommand{\hmwkClass}{Cloud\ Computing}
\newcommand{\hmwkAuthorName}{Shumin\ Guo}

% Setup the header and footer
\pagestyle{fancy}                                                       %
\lhead{\hmwkAuthorName}                                                 %
\chead{\hmwkClass\ \hmwkTitle}  %
\rhead{Page\ \thepage\ of\ \pageref{LastPage}} %
\lfoot{}                                                      %
\cfoot{}                                                                %
\rfoot{}                          %
\renewcommand\headrulewidth{0.4pt}                                      %
\renewcommand\footrulewidth{0.4pt}                                      %
\titleformat{\section}{\large\bfseries}{\thesection}{1em}{}
%%%%%%%%%%%%%%%%%%%%%%%%%%%%%%%%%%%%%%%%%%%%%%%%%%%%%%%%%%%%%
% Make title
\title{\textmd{\textbf{\hmwkClass:\
      \hmwkTitle}}\\\normalsize\small{Due\ Date:\
    \hmwkDueDate}\\} 
\date{\today}
\author{\textbf{\hmwkAuthorName}}
%%%%%%%%%%%%%%%%%%%%%%%%%%%%%%%%%%%%%%%%%%%%%%%%%%%%%%%%%%%%%

\begin{document}
\maketitle

\section*{Bigtable: A Distributed Storage System for
  Structured Data.}
With the arriaval web2.0, much more data is produced by online
computer systems, which is usually calculated in petabytes. This poses
a great challenge to those traditional database 
systems which are not very efficient for handling such large size
data. And this problem must be even more urgent in google, as it's core
business model is based on data. Either it be crawed
webpage data, web access log data or large volumes of picture and
streaming content data, the size is huge. All the requirements
promoted the design and implementation of the google's data hosting
system - Bigtable. 

Bigtable is used to store structured data. And in many ways it
resembles a database system, and shared many implementation strategies
with databases. But it also has a lot of new features compared
with traditional databases. It is designed to run on commodity
machines and can scale to petabytes of data and thousands of
machines. It provides clients with a simple data model that supports
dynamic control over data layout and format, and allows clients to
reason about the locality properties of the data represented in the
underlying storage. So, clients can control the locality of their data
through carefully choose their data schemas. Also, clients can use
Bigtable parameter to dynamically control whether to serve data in 
memory or from disk. 

Unlike the flat table structure of traditional data model, Bigtable
uses a very simple data model that not only considers the underlying
structure of data, but also tracks different versions of data. This is
a unique requirement for web page data as it changes over
time. To make it simple, bigtable is just like a map. The key/index of
this map is generated from strings of rows, columns and timestamps
of data. And the data of this map is uninterpreted bytes no matter
what type of data it is represented in the application. Bigtable
maintains data in lexicograhic order by row key, and the row range for
a table is dynamically partitioned. Each row range is a tablet, which
is the unit of distribution and load balancing. Usually, the row keys
are the URLs associated with the data content. Columns are properties
with the row content, such as anchor pointing to the content, or the
content itself. And the columns are usually grouped into column
families, and a column family can contain multiple family
qualifiers. Such as anchor is a family, and a URL address can be a
qualifier of this column family. This feature brings a lot of
benefits when accessing the contents of table. But as we can easily
infer that data in Bigtable is very sparse. Bigtable API provides some
other features of data manipulation such as row based transaction and
execution of client-supplied scripts in the address space of the
server. These features make Bigtable a full-fledged data storage
choice. 

Bigtable depends on a cluster management system for scheduling jobs,
managing resources on shared machines, dealing with machine failures,
and monitoring machine status. The Google SSTable file format is used
internally to store Bigtable data. An SSTable provides a persistent,
ordered and immutable map from keys to values, and internally, it contains
a sequence of blocks along with a block index to locate blocks. The
index will be loaded into memory when SSTable is opened. And a binary
search method can be used to search the Sorted String Table, and thus
greatly impoved the performace by eliminating I/O operations. Bigtable
relies on a highly-available and persistent distributed lock service
named Chubby. Chubby provides a namespace that consists of directories
and small files like UNIX file system. And each directory and file can
be used as a lock, and reads from and writes to a file are
atomic. Chubby service greatly simplified the management of locks
within a file system and this centralized lock management brings a lot
of benefits as compared with the ad-hoc mode lock management. Still,
Chubby's serving as a namespace of files make its no need to provide
dedicated name server. 

Bigtable has three major components: a library linked
into every client, one master server, and many tablet servers. Tablet
servers can be dynamically added from a cluster to accormmodate
changes in workloads. At times of data transfer, clients communicate
directly with tablet servers for reads and writes. This makes the
master server lightly loaded. And this communication schema can
improve the availability and throughput of the whole system. 

Bigtable uses a three-level hierarchy to store tablet location
information. Chubby contains the location of the root tablet, the root
tablet contains the location of all tablets in a special METADATA
table. And each METADATA table contains the location of a set of user
tablets. Each tablet is assigned to one tablet server at a time. The
master will keep track of the set of live tablet servers, and the
assignment of unassignment of current tablets. And the master
periordically checks with tablet servers for the status of its locks
so that it can keep the timely status of all the tablet servers and
tablet assignments. When a master server is started, it will first get
a unique master lock on Chubby, then it will poll the status of tablet
servers that are alive and finally it will communicate with these
servers to find tablets that thses servers are serving.  Bigtable uses
memtable to aggregate the log and commit information to reduce I/O
operations. For read and write operations, a tablet server need to
contact with Chubby to check the well-formedness of tablet and
authorization information. 

Besides the basic functionality of data storage and serving, there are
a few refinements to the Bigtable system. Clients can use locality
groups to group data with high correlation, and clients can choose
whether to use comppression to the data. The row name based clustering
makes a very good compression ratio possible, and even higher for
multiple versions of same data in Bigtable. 

In summary, the novel data model, the Chubby lock service and the
dynamic management of tablet servers and tablets can help Bigtable
achive the design goal of wide applicability, scalability, high
performance and high availability. And the successful application of
Bigtable to various production storage is the biggest proof of its
succss. And the consideration of client experience makes it one of
good choice of cloud data hosting. 

\end{document}