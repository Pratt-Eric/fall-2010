\documentclass{article}
% Change "article" to "report" to get rid of page number on title page
\usepackage{amsmath,amsfonts,amsthm,amssymb}
\usepackage{algorithmic,algorithm}
\usepackage{setspace}
\usepackage{Tabbing}
\usepackage{fancyhdr}
\usepackage{lastpage}
\usepackage{extramarks}
\usepackage{chngpage}
\usepackage{soul,color}
\usepackage{graphicx,float,wrapfig}

\usepackage{listings}
\usepackage{color}
\usepackage{textcomp}
\definecolor{listinggray}{gray}{0.9}
\definecolor{lbcolor}{rgb}{0.9,0.9,0.9}
\lstset{
	%backgroundcolor=\color{lbcolor},
	tabsize=4,
	rulecolor=,
	language=c++,
        basicstyle=\setstretch{1},
        upquote=true,
        aboveskip={\baselineskip},
        columns=fixed,
        showstringspaces=false,
        extendedchars=true,
        breaklines=true,
        prebreak = \raisebox{0ex}[0ex][0ex]{\ensuremath{\hookleftarrow}},
        %frame=single,
        showtabs=false,
        showspaces=false,
        showstringspaces=false,
        identifierstyle=\ttfamily,
        keywordstyle=\color[rgb]{0,0,1},
        commentstyle=\color[rgb]{0.133,0.545,0.133},
        stringstyle=\color[rgb]{0.627,0.126,0.941},
}
% In case you need to adjust margins:
\topmargin=-0.45in      %
\evensidemargin=0in     %
\oddsidemargin=0in      %
\textwidth=7.0in        %
\textheight=9.2in       %
\headsep=0.25in         %

% Homework Specific Information
\newcommand{\hmwkTitle}{Homework\ \#1}
\newcommand{\hmwkDueDate}{Sep.\ 23,\ 2010}
\newcommand{\hmwkClass}{Data Structure}
\newcommand{\hmwkClassTime}{TR\ 4:10-5:25pm}
\newcommand{\hmwkClassInstructor}{Meilin\ Liu}
\newcommand{\hmwkAuthorName}{Shumin\ Guo}

% Setup the header and footer
\pagestyle{fancy}                                                       %
\lhead{\hmwkAuthorName}                                                 %
\chead{\hmwkClass\ - \hmwkTitle}  %
\rhead{Page\ \thepage\ of\ \pageref{LastPage}}                          %
\lfoot{\lastxmark}                                                      %
\cfoot{}                                                                %
\rfoot{}                          %
\renewcommand\headrulewidth{0.4pt}                                      %
%\renewcommand\footrulewidth{0.2pt}                                     %

% This is used to trace down (pin point) problems
% in latexing a document:
%\tracingall

%%%%%%%%%%%%%%%%%%%%%%%%%%%%%%%%%%%%%%%%%%%%%%%%%%%%%%%%%%%%%
% Some tools
\newcommand{\enterProblemHeader}[1]{\nobreak\extramarks{#1}{#1 continued on next page\ldots}\nobreak%
                                    \nobreak\extramarks{#1 (continued)}{#1 continued on next page\ldots}\nobreak}%
\newcommand{\exitProblemHeader}[1]{\nobreak\extramarks{#1 (continued)}{#1 continued on next page\ldots}\nobreak%
                                   \nobreak\extramarks{#1}{}\nobreak}%

\newlength{\labelLength}
\newcommand{\labelAnswer}[2]
  {\settowidth{\labelLength}{#1}%
   \addtolength{\labelLength}{0.25in}%
   \changetext{}{-\labelLength}{}{}{}%
   \noindent\fbox{\begin{minipage}[c]{\columnwidth}#2\end{minipage}}%
   \marginpar{\fbox{#1}}%

   % We put the blank space above in order to make sure this
   % \marginpar gets correctly placed.
   \changetext{}{+\labelLength}{}{}{}}%

\setcounter{secnumdepth}{0}
\newcommand{\homeworkProblemName}{}%
\newcounter{homeworkProblemCounter}%
\newenvironment{homeworkProblem}[1][]%
  {\stepcounter{homeworkProblemCounter}%
   \renewcommand{\homeworkProblemName}{#1}%
   \section{\homeworkProblemName}%
   \enterProblemHeader{\homeworkProblemName}}%
  {\exitProblemHeader{\homeworkProblemName}}%

\newcommand{\problemAnswer}[1]
  {\noindent\fbox{\begin{minipage}{\columnwidth}#1\end{minipage}}}%

\newcommand{\problemLAnswer}[1]
  {\labelAnswer{\homeworkProblemName}{#1}}

\newcommand{\homeworkSectionName}{}%
\newlength{\homeworkSectionLabelLength}{}%
\newenvironment{homeworkSection}[1]%
  {% We put this space here to make sure we're not connected to the above.
   % Otherwise the changetext can do funny things to the other margin

   \renewcommand{\homeworkSectionName}{#1}%
   \settowidth{\homeworkSectionLabelLength}{\homeworkSectionName}%
   \addtolength{\homeworkSectionLabelLength}{0.25in}%
   \changetext{}{-\homeworkSectionLabelLength}{}{}{}%
   \subsection{\homeworkSectionName}%
   \enterProblemHeader{\homeworkProblemName\ [\homeworkSectionName]}}%
  {\enterProblemHeader{\homeworkProblemName}%

   % We put the blank space above in order to make sure this margin
   % change doesn't happen too soon (otherwise \sectionAnswer's can
   % get ugly about their \marginpar placement.
   \changetext{}{+\homeworkSectionLabelLength}{}{}{}}%

\newcommand{\sectionAnswer}[1]
  {% We put this space here to make sure we're disconnected from the previous
   % passage

   \noindent\fbox{\begin{minipage}[c]{\columnwidth}#1\end{minipage}}%
   \enterProblemHeader{\homeworkProblemName}\exitProblemHeader{\homeworkProblemName}%
   \marginpar{\fbox{\homeworkSectionName}}%

   % We put the blank space above in order to make sure this
   % \marginpar gets correctly placed.
   }%

%%%%%%%%%%%%%%%%%%%%%%%%%%%%%%%%%%%%%%%%%%%%%%%%%%%%%%%%%%%%%


%%%%%%%%%%%%%%%%%%%%%%%%%%%%%%%%%%%%%%%%%%%%%%%%%%%%%%%%%%%%%
% Make title
\title{\textmd{\textbf{\hmwkClass:\
      \hmwkTitle}}\\\normalsize\small{Due\ Date:\
    \hmwkDueDate}\\} 
\date{\today}
\author{\textbf{\hmwkAuthorName}}
%%%%%%%%%%%%%%%%%%%%%%%%%%%%%%%%%%%%%%%%%%%%%%%%%%%%%%%%%%%%%

\begin{document}
% \begin{spacing}{1.1}
\maketitle
% \newpage
% % Uncomment the \tableofcontents and \newpage lines to get a Contents page
% % Uncomment the \setcounter line as well if you do NOT want subsections
% %       listed in Contents
% %\setcounter{tocdepth}{1}
% \tableofcontents
% \newpage

% % When problems are long, it may be desirable to put a \newpage or a
% % \clearpage before each homeworkProblem environment

% \clearpage
\begin{homeworkProblem}
1. (10 points) Suppose that computer A executes 10 billion
instructions per second, and computer B executes 10 million
instructions per second, i.e, Computer A is 1000 times faster than
computer B in raw computing power. Suppose an expert programmer
implements insertion sort in machine language for computer A, and the
resulting code requires $5*n^2$ instructions to sort n numbers. Suppose
an average programmer implements merge sort, using a high-level
language(for computer B??), with the resulting code taking
$20*n*log_2n$ instruction. How long does computer A and computer B
take to sort 10 million numbers respetively?\\

\problemAnswer{
For computer A: time = $\frac{5\times(10^7)^2}{10^{10}}$ = $5\times 10^4$(s) = 50000(s) \\
For computer B: time = $\frac{20\times 10^7\times log_210^7}{10^7}$ =
$20\times log_210^7 \approx 465.07(s)$.\\

According to the result above, we can see that although computer A is
1000 times faster than computer B, it is using a less efficient
algorithm. So, as the result tells, the real computing time for large
input data for computer A is much longer than computer B. \\ 
And we can have the conclusion that the efficiency of algorithm is
more of concern as compared with computing power for a particular
computing problem.
}
\end{homeworkProblem}

\begin{homeworkProblem}
2. (20 Points) Be sure to show your work for each of the following
problems. \\
\begin{itemize}
\item Suppose that a particular algorithm has time complexity $T(n) =
5*2^n$, and that execution of the algorithm on a particular machine
takes T seconds for n inputs. Now, suppose you are presented with a
machine that is 64 times as fast as your current machine. How many
inputs can you process on you new machine in T seconds? \\

\problemLAnswer{
According to the question description, we have:\\
$T = \frac{5\times 2^n}{P}$~ (P is the computing power with unit instructions /
second) \\
$\Rightarrow P = 5 \times \frac{2^n}{T}$ \\
$\Rightarrow P' = 64 \times P = \frac{64 \times 5 \times 2^n}{T}$ \\
And for the new machine, we have: \\
$T = \frac{5 \times 2^{n'}}{P'} = \frac{5 \times 2^{n'}}{\frac{64
\times 5 \times 2^n}{T}}$ \\  
$\Rightarrow T = T \times \frac{2^{n'-n}}{64}$ \\ 
$\Rightarrow 2^{n'-n} = 64$\\
$\Rightarrow n'-n = 6 \Rightarrow n'=n+6$ \\
So, the new machine can process n+6 instructions within T seconds. 
}
\item Suppose that a particular algorithm has time complexity $T(n) =
2*n^3$, and that execution of the algorithm on a particular machine
takes T seconds for n inputs. Now, suppose you are presented with a
machine that is 64 times as fast as your current machine. How many
inputs can you process on you new machine in T seconds? \\

\problemLAnswer{
Similar to the question above, we have: \\
$T = 2 \times \frac{n^3}{P}\Rightarrow P = 2 \times \frac{n^3}{T}$ \\
And for the new machine, we have: \\
$P' = 64 \times P = \frac{64 \times 2 \times n^3}{T}
\Rightarrow T = \frac{2 \times n'^{3}}{P'} = \frac{2 \times n'^{3}}{\frac{64
\times 2 \times n^3}{T}}
\Rightarrow n'^3 = 64\times n^3 = (4n)^3
\Rightarrow n' = 4n$ \\ 
So, the new machine can process 4n instruction within T seconds. 
}
\item Suppose that a particular algorithm has time complexity $T(n) =
16n$, and that execution of the algorithm on a particular machine
takes T seconds for n inputs. Now, suppose you are presented with a
machine that is 64 times as fast as your current machine. How many
inputs can you process on you new machine in T seconds? \\

\problemLAnswer{
Similar to the question above, we have: \\
$T = \frac{16n}{P}
\Rightarrow P = 16 \times \frac{n}{T}$.

And for the new machine, we have: \\
$P' = 64 \times P = \frac{64 \times 16 \times n}{T}
\Rightarrow T = \frac{16 \times n'}{P'} = \frac{16 \times n'}{\frac{64
\times 16 \times n}{T}}  
\Rightarrow n' = 64n$.

So, the new machine can process 64n instruction within T seconds.
}
\end{itemize}

\end{homeworkProblem}

\begin{homeworkProblem}
3. (10 Points) Arrange the following expressions by growth, from
slowest to fastest \\

$4n^2,~ log_3n,~ n!,~ 3^n,~ 8n,~ 25,~ log_2n,~ n^{\frac{2}{3}}$ \\

\problemLAnswer{
$25 < log_3n < log_2n < n^{\frac{2}{3}} < 8n < 4n^2 < 3^n < n!$
}
\end{homeworkProblem}

\begin{homeworkProblem}
4. (20 points) Using the definition of Big-O, show that: 

\begin{itemize}
\item $2^{100} = O(1)$ \\

\problemLAnswer{
Let $f(n) = 2^{100}$, $g(n) = 1$ And\\
If let constant $C = 2^{100} + 1$ and $n_0 = 0$, \\ 
we have: $\forall~ n \geq n_0$, $0\leq f(n) \leq C.g(n)$ \\

Thus, according to definition of Big-O, $\exists~ C$ and for all $n \geq
n_0$, $0\leq f(n) \leq C.g(n)$ \\
So, $f(n) = O(g(n)) = O(1)$, which is $2^{100} = O(1)$.
}

\item $8n^2 + 15n + 6 = O(n^2)$ \\

\problemLAnswer{
Let $f(n) = 8n^2 + 15n + 6$ and $g(n) = n^2$ \\
$\because~ 15n + 6 \leq n^2$~ when $n \geq 16$ \\
$\therefore~ 8n^2 + 15n + 6 \leq 8n^2 + n^2 = 9n^2$ \\
Let constant $C = 9$ and $n_0 = 16$, when $n \geq n_0$ we have
inequation $0 \leq f(n) \leq C.g(n)$. \\
And thus: $f(n) = O(g(n)) = O(n^2)$, which is $8n^2 + 15n + 6 = O(n^2)$.
}

\item $3\times (2^n + n^2) = O(2^n)$

\problemLAnswer{
Let $f(n) = 3\times (2^n + n^2)$ and $g(n) = 2^n$ \\
$\because~  3\times n^2 \leq 2^n$~ when $n \geq 8$ \\
$\therefore~ 3.n^2 + 3.2^n \leq 2^n + 3.2^n = 4.2^n$ when $n\geq 8$ \\
Let constant $C = 4$ and $n_0 = 8$, when $n \geq n_0$ we have \\
$0 \leq f(n) \leq C.g(n)$, So $f(n) = O(g(n)) = O(n^2)$, which is
$3\times (2^n + n^2) = O(2^n)$. 
}
\end{itemize}
\end{homeworkProblem}

\begin{homeworkProblem}
5. (5 points) Express the function $\frac{n^3}{1000} + 10n^2 + 100n +
3$ in terms of $\Theta$−notation. \\

\problemLAnswer{
Let $f(n) = \frac{n^3}{1000} + 10n^2 + 100n + 3$ and $g(n) = n^3$, \\
$\because$~ we have $10.n^2 + 100n + 3 \leq n^3$, when $n\geq 17$\\
$\therefore~ \frac{n^3}{1000} + 10.n^2 + 100n + 3 \leq
\frac{n^3}{1000} + n^3 = \frac{1001.n^3}{1000}$ \\ 
$\Rightarrow \exists~ C_1 = \frac{1001}{1000}$~ and $n_{01} = 17$~ when $\forall n \geq
n_{01}$~ inequality $0\leq f(n) \leq C_1.g(n)$~ applies. \\
Similarly, when $C_2 = \frac{1}{1000}$ and $\forall~ n\geq n_{02}=0$ we
have $0\leq C_2g(n) \leq f(n)$.  \\
So using the big-$\Theta$~notation we can get: $\forall~ n \geq n_{01}
> n_{02},~ 0\leq C_2.g(n)\leq f(n) \leq C_1.g(n)$, where $C_1 =
\frac{1001}{1000}$ and $C_2 = \frac{1}{1000}$.  
}
\end{homeworkProblem}

%% 6
\begin{homeworkProblem}
6. (10 points) For each of the following functions, either $f(n)$ is in
$O(g(n))$, $f(n)$ is in $\Omega(g(n))$, or $f(n)$ is in
$\Theta(g(n))$. For each pair shown in the following table, determine
which relationship is correct: \\
\begin{table}[h]
  \begin{center}   
    \begin{tabular}{| l | c | r |} 
      \hline $f(n)$ & $g(n)$ & $f(n)=(X)g(n)$ \\
      \hline $logn^2$ & $logn + 5$ & $f(n) = \Theta(g(n))$ \\ 
      \hline $nlogn + n$ & $logn$ & $f(n) = \Omega(g(n))$ \\
      \hline $log^2n$ & $logn$ & $f(n) = \Omega(g(n))$ \\
      \hline 10 & $log10$ & $f(n) = \Theta(g(n))$ \\
      \hline $2^n$ & $10n^2$ & $f(n) = \Omega(g(n))$ \\
      \hline $2^n$ & $3^n$ & $f(n) = O(g(n))$ \\
      \hline 
    \end{tabular} 
    % \hfill{}
    \caption{Relationship of pair of functions.\label{tb:tablename}}
  \end{center}
  \vspace{-20pt}
\end{table}
\end{homeworkProblem}

% number 7
\begin{homeworkProblem}
7. (15 points) For each of the following sums, determine a closed-form
solution

\begin{itemize}
\item $\sum_{i=1}^n(6i + 5)$

\problemLAnswer{
Let $a_i = 6i + 5$ \\
$\Rightarrow a_1 = 6 + 5 = 11$ and $a_n = 6n + 5$, so \\
$\sum_{i=1}^na_i = \frac{(a_1+a_n).n}{2} = 3n^2 + 8n$
}

\item $\sum_{i=1}^7(2^i + 4i + 7)$

\problemLAnswer{
$= \sum_{i=1}^72^i + \sum_{i=1}^74i + \sum_{i=1}^77$ \\
$= \sum_{i=0}^72^i - 1 + \sum_{i=1}^74i + \sum_{i=1}^77$ \\
$= \frac{2^{7+1}-1}{2-1} - 1 + \frac{(4+28)\times 7}{2} + 7\times 7$ \\
$= 254 + 112 + 49$ \\
$= 415$
}

\end{itemize}

\end{homeworkProblem}

\begin{homeworkProblem}
8. (10 points) Express the running time for each of the following
fragments of code, as a function of N:

\begin{itemize}
\item 
  \begin{lstlisting}
    for(i=1; i<=N; i++){
      for(j=1; j<=N; j++){
        cout << i << j << endl;
      }
    }
  \end{lstlisting}
  T(N) = $\sum_{i=1}^NN = N\times N = N^2$, \\
  $\Rightarrow~ 0 \leq \frac{1}{2}N^2 \leq N^2 \leq 2N^2$, $\forall N
  \geq 0$ \\
  $\Rightarrow~ \forall N \geq N_0 = 0$ and $C_1 = 2$ and $C_2 =
  \frac{1}{2}$, we have $0 \leq C_2N^2 \leq N^2 \leq C_1N^2$ \\

  And according to the big-$\Theta$~ definition: \\
  \textbf{$\Rightarrow~$T(N) = $\Theta(N^2)$}

\item 
  \begin{lstlisting}
    for(i=1; i<=N; i++){
      for(j=1; j<=i; j++){
        cout << i << j << endl;
      }
    }
  \end{lstlisting}
  T(N) = $\sum_{i=1}^Ni = 1 + 2 + 3 + \ldots + N = \frac{(1+N)N}{2} =
  \frac{1}{2}(N^2 + N)$, \\ 
  and using similar method to the above question, we will have : \\ 
  $\forall N\geq N_0 = 0$, and $C_1 = 1, C_2 = \frac{1}{2}$, $0\leq
  C_2N^2 \leq T(N) \leq C_1N^2$ \\    
  \textbf{$\Rightarrow~$T(N) = $\Theta(N^2)$}
\end{itemize}
\end{homeworkProblem}

\begin{homeworkProblem}
9. Give an algorithm in pseudo-code to reverse a singly linked list of
n elements. The algorithm should rely on constant additional storage
(only a few variables, no additional linked lists, arrays, etc.) 

\begin{center}
  \begin{algorithm}[H]
    \caption{reverseList(pHead, n)\label{alg:reverse_str}}
    \begin{algorithmic}[1]
      \STATE{\textbf{Input:}Linked List with n elements ($n\geq~0$)}
      \STATE{\textbf{Output:}Reversed List}
      \STATE pTail $\leftarrow$ TailofList
      \STATE pTmp $\leftarrow$ NULL
      \STATE i $\leftarrow$ 0
      \WHILE {$i \leq n-1$}
      \STATE $i \leftarrow i + 1$
      \STATE pTmp $\leftarrow$ pHead
      \STATE pHead $\leftarrow$ pHead.next
      \STATE pTmp.next $\leftarrow$ NULL
      \STATE pTail.next $\leftarrow$ pTmp
      \STATE pTail $\leftarrow$ pTmp
      \ENDWHILE
      \RETURN TRUE
    \end{algorithmic}
  \end{algorithm}
\end{center}
%}
\end{homeworkProblem}

%\end{spacing}
\end{document}

%%%%%%%%%%%%%%%%%%%%%%%%%%%%%%%%%%%%%%%%%%%%%%%%%%%%%%%%%%%%%
