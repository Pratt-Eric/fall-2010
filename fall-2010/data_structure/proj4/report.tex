\documentclass{article}
\usepackage{amsmath,amsfonts,amsthm,amssymb}
\usepackage{fancyhdr}
\usepackage{lastpage}
\usepackage{soul,color}
\usepackage{graphicx,float,wrapfig}
\usepackage{hyperref}
\usepackage[scriptsize]{subfigure}
\usepackage{enumerate}
\usepackage{multicol}
\usepackage[font=small,labelfont=bf]{caption}
\usepackage{listings}
% In case you need to adjust margins:
\topmargin=-0.45in      %
\evensidemargin=-0.5in     %
\oddsidemargin=-.5in      %
\textwidth=7.0in        %
\textheight=9.2in       %
\headsep=0.25in         %

\begin{document}
\begin{center}
\textbf{{\Large CS 400/600 – Programming Assignment \#4 –Hash Table}} \\
\textsc{Shumin Guo} \\
\small{Due Date: Nov. 11th, 2010}
\end{center}

\section*{Report}
Along with your working source code (.cpp and .h files), you should
submit a brief ($<$ 5 pages) report describing how the closed hash
function performed as a function of the load factor of the hash table,
alpha. You should perform a variety of searches, and compare the
number of nodes searched for each index.  You should address questions
such as: 
\begin{itemize}
\item How does the hash perform in terms of search efficiency when the hash
  table is (1) nearly empty (the alpha is very small, like less than
  0.1),  (2) moderately full (alpha is around 0.5),  (3) very full
  (alpha is larger than 0.8). 

\begin{table}[H]
  \begin{center}
    \begin{tabular}{c|c|c|c} 
      \hline
      \multicolumn{2}{c}{Double Probing} &
      \multicolumn{2}{c}{Random Probing} \\
      \hline
      $\alpha$ & Probes & $\alpha$ & Probes \\
      \hline
      0.015274 & 0.017 & 0.015274 & 0.017 \\
      0.0458221 & 0.049 & 0.0458221 & 0.052 \\
      0.0763702 & 0.065	& 0.0763702 & 0.08 \\
      0.106918 & 0.116 & 0.106918 & 0.101 \\
      0.137466 & 0.148 & 0.137466 & 0.164 \\
      0.168015 & 0.205 & 0.168015 & 0.173 \\
      0.198563 & 0.278 & 0.198563 & 0.247 \\
      0.229111 & 0.338 & 0.229111 & 0.294 \\
      0.259659 & 0.367 & 0.259659 & 0.356 \\
      0.290207 & 0.451 & 0.290207 & 0.394 \\
      0.320755 & 0.461 & 0.320755 & 0.483 \\
      0.351303 & 0.636 & 0.351303 & 0.54 \\
      0.381851 & 0.677 & 0.381851 & 0.601 \\
      0.412399 & 0.808 & 0.412399 & 0.721 \\
      0.442947 & 0.907 & 0.442947 & 0.79 \\
      0.473495 & 1.017 & 0.473495 & 0.88 \\
      0.504044 & 1.154 & 0.504044 & 0.995 \\
      0.534592 & 1.185 & 0.534592 & 1.075 \\
      0.56514 & 1.365 & 0.56514 & 1.313 \\
      0.595688 & 1.735 & 0.595688 & 1.413 \\
      0.626236 & 1.843 & 0.626236 & 1.745 \\
      0.656784 & 2.184 & 0.656784 & 1.983 \\
      0.687332 & 2.423 & 0.687332 & 2.287 \\
      0.71788 & 2.8 & 0.71788 & 2.457 \\
      0.748428 & 3.415 & 0.748428 & 3.033 \\
      0.778976 & 3.959 & 0.778976 & 3.667 \\
      0.809525 & 4.665 & 0.809525 & 4.193 \\
      0.840073 & 5.275 & 0.840073 & 5.227 \\
      0.870621 & 6.792 & 0.870621 & 6.798 \\
      0.901169 & 9.986 & 0.901169 & 9.422 \\
      0.931717 & 14.752 & 0.931717 & 14.45 \\
      0.962265 & 28.271 & 0.962265 & 25.891 \\
      \hline
    \end{tabular}
    \caption{Load Factor $\alpha$ VS. Probes.\label{tbl:effi}} 
    \vspace{-15pt}
  \end{center}
\end{table}

\begin{figure}[H]
\vspace{-15pt}
  \begin{center}
    \includegraphics[scale=0.5]{alpha-collision}
    \caption{Average Search Efficiency for Double and Pseudo-random
      Probing Hashing.\label{fig:effi}}
  \end{center}
\vspace{-10pt}
\end {figure}

From Table \ref{tbl:effi} and Figure \ref{fig:effi} we can see that
double hashing and random hashing has identical search
efficiency. When the Hash Table is nearly empty or when $\alpha$ is
very small close to zero, The average probes for search and insert is
minimal, it is far blow one on average. When the Hash Table is
relatively full, that is when $\alpha$ is close to 0.5, the average
number of probes for insert and search is relatively increased as
compared with the previous case, but the number is still very low on
average. When the Hash Table is becoming full, the number of probes
start to increase far when $\alpha$ is above 0.8, and the worst case
began to happen when $\alpha$ is above 0.9. The average number of
collisions skyrocketed to over 20 and even more than 25 for random
probing. In this case, the Hash Table is heavily overloaded.

\item Give some intuitive comments about the performance of your hash table
  regarding the load factor, $\alpha$.\\

  The data for the above question is collected by doing several
  experiments for different load factor $\alpha$. So, it can refect the
  time it takes to insert an element or find an element in the Hash
  Table. I experienced some time variations with different load
  factor. When the factor is very low, it is very fast to search/insert
  element, it is almost constant time operation, which clearly agrees
  with the result in the table and the figure. As the load factor
  increases with more and more elements inserted into the hash table,it
  takes a little longer time to do the same operations. And when load
  factor becomes very close to one. It takes much longer time than the
  previous scenarios, which mean more collisions are happening. This
  intuitive experience together with the concrete experiment data
  teaches us that the load factor should be kept at a relatively low
  level for it to work efficiently(constant time operation). Usually the
  load factor should be around 0.5 when 1 probes happen on average. When
  the load factor is too close to one, the hash table will lose its
  competativeness as compared with other data structures.
\end{itemize}

Your report should be supported with data in the form of tables,
charts, graphs, etc. 

Make sure to base your report on average results over many searches,
and not just one search. 

\end{document}

%%%%%%%%%%%%%%%%%%%%%%%%%%%%%%%%%%%%%%%%%%%%%%%%%%%%%%%%%%%%%
