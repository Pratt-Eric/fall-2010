\documentclass{article}
% Change "article" to "report" to get rid of page number on title page
\usepackage{amsmath,amsfonts,amsthm,amssymb}
\usepackage{setspace}
\usepackage{Tabbing}
\usepackage{fancyhdr}
\usepackage{lastpage}
\usepackage{extramarks}
\usepackage{chngpage}
\usepackage{soul,color}
\usepackage{graphicx,float,wrapfig}

% In case you need to adjust margins:
\topmargin=-0.45in      %
\evensidemargin=0in     %
\oddsidemargin=0in      %
\textwidth=6.5in        %
\textheight=9.0in       %
\headsep=0.25in         %

% Homework Specific Information
\newcommand{\hmwkTitle}{Project\ \#0}
\newcommand{\hmwkDueDate}{Monday,\ Sep.\ 13,\ 2010}
\newcommand{\hmwkClass}{CS-699-01/Cloud\ Computing}
\newcommand{\hmwkClassTime}{TR/8:00pm-9:15pm}
\newcommand{\hmwkClassInstructor}{Keke Chen}
\newcommand{\hmwkAuthorName}{Shumin\ Guo}

% Setup the header and footer
\pagestyle{fancy}                                                       %
\lhead{\hmwkAuthorName}                                                 %
\chead{\hmwkClass\ (\hmwkClassInstructor\ \hmwkClassTime): \hmwkTitle}  %
\rhead{\firstxmark}                                                     %
\lfoot{\lastxmark}                                                      %
\cfoot{}                                                                %
\rfoot{Page\ \thepage\ of\ \pageref{LastPage}}                          %
\renewcommand\headrulewidth{0.4pt}                                      %
\renewcommand\footrulewidth{0.4pt}                                      %

% This is used to trace down (pin point) problems
% in latexing a document:
%\tracingall

%%%%%%%%%%%%%%%%%%%%%%%%%%%%%%%%%%%%%%%%%%%%%%%%%%%%%%%%%%%%%
% Some tools
\newcommand{\enterProblemHeader}[1]{\nobreak\extramarks{#1}{#1 continued on next page\ldots}\nobreak%
                                    \nobreak\extramarks{#1 (continued)}{#1 continued on next page\ldots}\nobreak}%
\newcommand{\exitProblemHeader}[1]{\nobreak\extramarks{#1 (continued)}{#1 continued on next page\ldots}\nobreak%
                                   \nobreak\extramarks{#1}{}\nobreak}%

\newlength{\labelLength}
\newcommand{\labelAnswer}[2]
  {\settowidth{\labelLength}{#1}%
   \addtolength{\labelLength}{0.25in}%
   \changetext{}{-\labelLength}{}{}{}%
   \noindent\fbox{\begin{minipage}[c]{\columnwidth}#2\end{minipage}}%
   \marginpar{\fbox{#1}}%

   % We put the blank space above in order to make sure this
   % \marginpar gets correctly placed.
   \changetext{}{+\labelLength}{}{}{}}%

\setcounter{secnumdepth}{0}
\newcommand{\homeworkProblemName}{}%
\newcounter{homeworkProblemCounter}%
\newenvironment{homeworkProblem}[1]%[Project \arabic{homeworkProblemCounter}]%
  {\stepcounter{homeworkProblemCounter}%
   \renewcommand{\homeworkProblemName}{#1}%
   \section{\homeworkProblemName}%
   \enterProblemHeader{\homeworkProblemName}}%
  {\exitProblemHeader{\homeworkProblemName}}%

\newcommand{\problemAnswer}[1]
  {\noindent\fbox{\begin{minipage}[c]{\columnwidth}#1\end{minipage}}}%

\newcommand{\problemLAnswer}[1]
  {\labelAnswer{\homeworkProblemName}{#1}}

\newcommand{\homeworkSectionName}{}%
\newlength{\homeworkSectionLabelLength}{}%
\newenvironment{homeworkSection}[1]%
  {% We put this space here to make sure we're not connected to the above.
   % Otherwise the changetext can do funny things to the other margin

   \renewcommand{\homeworkSectionName}{#1}%
   \settowidth{\homeworkSectionLabelLength}{\homeworkSectionName}%
   \addtolength{\homeworkSectionLabelLength}{0.25in}%
   \changetext{}{-\homeworkSectionLabelLength}{}{}{}%
   \subsection{\homeworkSectionName}%
   \enterProblemHeader{\homeworkProblemName\ [\homeworkSectionName]}}%
  {\enterProblemHeader{\homeworkProblemName}%

   % We put the blank space above in order to make sure this margin
   % change doesn't happen too soon (otherwise \sectionAnswer's can
   % get ugly about their \marginpar placement.
   \changetext{}{+\homeworkSectionLabelLength}{}{}{}}%

\newcommand{\sectionAnswer}[1]
  {% We put this space here to make sure we're disconnected from the previous
   % passage

   \noindent\fbox{\begin{minipage}[c]{\columnwidth}#1\end{minipage}}%
   \enterProblemHeader{\homeworkProblemName}\exitProblemHeader{\homeworkProblemName}%
   \marginpar{\fbox{\homeworkSectionName}}%

   % We put the blank space above in order to make sure this
   % \marginpar gets correctly placed.
   }%

%%%%%%%%%%%%%%%%%%%%%%%%%%%%%%%%%%%%%%%%%%%%%%%%%%%%%%%%%%%%%


%%%%%%%%%%%%%%%%%%%%%%%%%%%%%%%%%%%%%%%%%%%%%%%%%%%%%%%%%%%%%
% Make title
\title{\vspace{2in}\textmd{\textbf{\hmwkClass\\ \hmwkTitle}}\\\normalsize\vspace{0.1in}\small{Due\ on\ \hmwkDueDate}\\\vspace{0.1in}\large{\textit{\hmwkClassInstructor\ \hmwkClassTime}}\vspace{3in}}
\date{}
\author{\textbf{\hmwkAuthorName}}
%%%%%%%%%%%%%%%%%%%%%%%%%%%%%%%%%%%%%%%%%%%%%%%%%%%%%%%%%%%%%

\begin{document}
\begin{spacing}{1.1}
\maketitle
\newpage
% Uncomment the \tableofcontents and \newpage lines to get a Contents page
% Uncomment the \setcounter line as well if you do NOT want subsections
%       listed in Contents
%\setcounter{tocdepth}{1}
%\tableofcontents
%\newpage

% When problems are long, it may be desirable to put a \newpage or a
% \clearpage before each homeworkProblem environment

\clearpage
\begin{homeworkProblem}{Some Understanding of Cloud Computing.}

In recent years, the buzz word "Cloud Computing" is attracting
people's attention. Companies like google, microsoft, amazon are
making great efforts to build their cloud infrastructure, even the
newly borned star company facebook is trying to join this group. Then,
what Cloud Computing is all about? In this short note I would like to
describe some of my own understanding about cloud computing. I will
try to describe my points with the following questions in mind: what
is cloud computing, why cloud computing can become the next
revolutionary computing model and current obstacles for the cloud
computing infrastructure. While Cloud Computing is not only computing
for businesses but also for individual person, here I will only focus
on the Cloud Computing for business customers rather than individuals
to whom it means differently.

Briefly speaking, Cloud Computing is utility computing, which means we
can use all kinds of computing resources the similiar way we use
electric and water. There is no contract between service providers and
customers, this pay-as-you-go business model is about to bring
customers great resilence and efficiency to their business as compared
to the traditional IT-department model. As we all know, companies will
have to build their own IT infrastructure for their business to
operate upon. For example, IT infrastructure is one of the most
critical part for a financial companies, so they need to have a
department which is responsible for maintaining their business
backbone. But, as a matter of fact, the resources including computing
and personel are underutilized for most of the time. On the other hand,
Cloud Computing makes an abstraction of the computing resources among
all these companies and move them to a central location called "Data
Center". All these companies can share these computing resources at
the same time or in a multiplexing way. This will greatly reduce the
cost as compared with the traditional model. Additionally, Cloud
Computing model will bring resilence to the resources that customers
use. One such case is facebook, the most famous social networking
online community, the burst feature of their website visit requests a
dynamic provisioning model that allocate different amount of
resources which Cloud Computing can provide. And the traditional model
can't respond fast enough to the growing of the visit volume. 

The world around us is evolving, to the direction of efficiency, so
does our computing model. Centralizing the computing resources to
those companies who are excel in doing this and liberating those
companies which does other business with IT as their
infrastructure is a proper description of what cloud computing brings
us. Everything seems natural. But except for the business
needs and business competition for cloud computing, there are other
enablers for this to happen at this proper time. They are the basis
for cloud computing to become a reality. That is the development of computer
hardware and software. With declining prices and improving computing
power, the computer hardware industry provides commodity computers for
use in the cloud. Virtualization techniques makes the dynamic
provisioning of hardware resource a reality and the high speed local
and wide area network makes it possible to delivery computing the
similar way as deliverying electric. 

Years ago, web2.0 and SOA is the preliminary model of the
cloud. Today, cloud computing can have even more features from
Infrastructre as a service(IaaS), Platform as a service(PaaS) to
Software as a service(SaaS). All these kind of features make it
possible to build their own business upon the cloud for today's
business. 

But, there is still a long way to go for all the world to be "Cloud
Computing". There are still a lot of obstacles which include privacy,
security, availability etc. Great efferts should be made both in the
industry and in the research world. But the practice of cloud
computing by such companies as Amazon, Microsoft and Google have
provided a model and industrial practice for the comminity to follow and
improve. Also, there are intermidiate models such as private cloud for
companies to adapt if such issues as privacy is a big concern. So,
thus far, we have the reason to believe that Cloud Computing will be
the next revolutionary computing model that will provide computing
resources in the CLOUD. 

\problemAnswer{}
\end{homeworkProblem}

\end{spacing}
\end{document}