\documentclass{article}
% Change "article" to "report" to get rid of page number on title page
\usepackage{amsmath,amsfonts,amsthm,amssymb}
\usepackage{algorithmic,algorithm}
\usepackage{setspace}
\usepackage{Tabbing}
\usepackage{fancyhdr}
\usepackage{lastpage}
\usepackage{extramarks}
\usepackage{chngpage}
\usepackage{soul,color}
\usepackage{ulem}
\usepackage{graphicx,float,wrapfig}
\usepackage{amsfonts}
\usepackage{pifont}
\usepackage{booktabs}
\usepackage{hyperref}
\usepackage{pstricks,pst-node,pst-tree}

\newcommand{\tickYes}{\checkmark}
\newcommand{\tickNo}{\hspace{1pt}\ding{55}}

% In case you need to adjust margins:
\topmargin=-0.45in      %
\evensidemargin=0in     %
\oddsidemargin=0in      %
\textwidth=6.8in        %
\textheight=9.4in       %
\headsep=0.25in         %

% Homework Specific Information
\newcommand{\hmwkTitle}{Homework\ \#4}
\newcommand{\hmwkDueDate}{Nov.\ 10,\ 2010}
\newcommand{\hmwkClass}{Data Mining}
\newcommand{\hmwkClassTime}{MW\ 4:10-5:25pm}
\newcommand{\hmwkClassInstructor}{Guozhu\ Dong}
\newcommand{\hmwkAuthorName}{Shumin\ Guo}

% Setup the header and footer
\pagestyle{fancy}                                                       %
\lhead{\hmwkAuthorName}                                                 %
\chead{\hmwkClass\ - \hmwkTitle}  %
\rhead{Page\ \thepage\ of\ \pageref{LastPage}}                          %
\lfoot{\lastxmark}                                                      %
\cfoot{}                                                                %
\rfoot{}                          %
\renewcommand\headrulewidth{0.4pt}                                      %
%\renewcommand\footrulewidth{0.4pt}                                     %

%%%%%%%%%%%%%%%%%%%%%%%%%%%%%%%%%%%%%%%%%%%%%%%%%%%%%%%%%%%%%
% Make title
\title{\textbf{\hmwkClass:\ 
      \hmwkTitle}\\\normalsize\small{Due\ Date:\
    \hmwkDueDate}} 
\date{\today}
\author{\textbf{\hmwkAuthorName}}
%%%%%%%%%%%%%%%%%%%%%%%%%%%%%%%%%%%%%%%%%%%%%%%%%%%%%%%%%%%%%

\begin{document}
% \begin{spacing}{1.1}
\maketitle

\begin{enumerate}
\item (a) Define the clustering problem, and (b) define the measure
  commonly used to evaluate the quality of clustering results. \\
\textbf{ANSWER:}\\ 
Clustering is the process of grouping the data into classes or
clusters, so that objects within a cluster have high similarity in
comparison to one another but are very dissimilar to objects in other
clusters. Dissimilarities are assessed based on the attribute values
describing the objects. Often, distance measures are used.\\
The quality of clustering is estimated using a cost function that
measures the average dissimilarity between an object and the
representative object of its cluster, such as square error.

\item Compare the advantages and disadvantages of (a)K-means and (b)
  K-medoids for clustering. (c) Discuss a main challenge to both
  K-means and K-medoids algorithms.\\
\textbf{ANSWER:} \\
Advantage:\\
The k-means algorithm takes the input parameter, k, and partitions a
set of n objects into k clusters so that the resulting intracluster
similarity is high but the intercluster similarity is low. The
algorithm attempts to determine k partitions that minimize the
square-error function. It works well when the clusters are compact
clouds that are rather well separated from one another. The method is
relatively scalable and efficient in processing large data sets
because the computational complexity of the algorithm is $O(nkt)$, where
n is the total number of objects, k is the number of clusters, and t
is the number of iterations. Normally, k$\ll$n and t$\ll$n. The method often
terminates at a local optimum. 

Disadvantage:\\
The necessity for users to specify k,
the number of clusters in advance can be seen as a disadvantage. The
k-means method is not suitable for discovering clusters with
nonconvex shapes or clusters of very different size.  Moreover, it is
sensitive to noise and outlier data points because a small number of
such data can substantially influence the mean value.

In order to reduce the sensitivity to outliers, K-medoids methods,
instead of taking the mean value of the objects in a cluster as a
reference point, pick actual objects to represent the clusters, using
one representative object per cluster.

The k-medoids method is more robust than k-means in the presence of
noise and outliers, because a medoid is less influenced by outliers or
other extreme values than a mean. However, its processing is more
costly than the k-means method. Both methods require the user to
specify k, the number of clusters.


\item Compare the COBWEB algorithm against other clustering algorithms
  such as K-means and K-mediods. \\
\textbf{ANSWER:} \\
COBWEB is a popular and simple method of incremental conceptual
clustering. Its input objects are described by categorical
attribute-value pairs. COBWEB creates a hierarchical clustering in
the form of a classification tree. While on the other hand, K-means
and K-mediods methods are used to cluster quantitative data based on
distance. \\ 
COBWEB uses a heuristic evaluation measure called category utility
to guide construction of the tree. It incrementally incorporates
objects into a classification tree in order to get the highest category
utility. And a new class can be created on the fly, which is one of
big difference between COBWEB and K-means and K-medoids methods. \\
COBWEB provides merging and splitting of classes based on category
utility, this allows COBWEB to be able to do bidirectional search. For
example, a merge can undo a previous split. While for K-means and
K-medoids methods, the clustering is usually unidirectional, which
means the cluster of a point is determined by the distance to the
cluster center. It might be very sensitive to the outliers in the
data. 

COBWEB has a number of limitations. First, it is based on the
assumption that probability distributions on separate attributes are
statistically independent of one another. This assumption is,
however, not always true because correlation between attributes often
exists. Moreover, the probability distribution representation of
clusters makes it quite expensive to update and store the
clusters. This is especially so when the attributes have a large
number of values because the time and space complexities depend not
only on the number of attributes, but also on the number of values for
each attribute. Furthermore, the classification tree is not
height-balanced for skewed input data, which may cause the time and
space complexity to degrade dramatically. And K-means and K-medoids
methods don't have such issues as considerations of probabilities and
independence. It only take into consideration of distance, but this
feature also renders them unproper for high dimensional data sets. 

\item Briefly discuss how recent algorithms deal with clustering for
  large amount of data, or nonsphere shaped clusters.  \\
\textbf{ANSWER:} \\

\begin{itemize}
\item Reducing the number of instances to be maintained
  while maintaining the distribution of regions/clusters.\\
  CLARA and CLARANS are representative Sampling Methods. 
  CLARA (Clustering LARge Applications) working on samples instead of
  the whole data, And CLARANS (Clustering Large Applications based on
  RANdomized Search).
  
\item Identifying relevant subspaces where clusters possibly exist.\\
  Grid: STING (STatistical INformation Grid) Statistical parameters of
  higher-level cells in the grid can easily be computed from those of
  lower-level cells.
  
\item Using summarized information to avoid repeated data access.\\
  BIRCH using Cluster Feature (CF) and CF tree \\
  CURE (Clustering Using REpresentitives) is another example. 
  
\item Taking advantage of the property of density. If the data is
  dense in a higher dimensional subspace, it should be dense in some
  lower dimensional subspaces. \\
  CLIQUE is a density-based method that can automatically find subspaces
  of the highest dimensionality such that high-density clusters exist in
  those subspaces.
\end{itemize}

\item Apply the K-means algorithm for the following 1-dimensional
  points and k=2:1,2,3,4,6,7,8,9. Use 1 and 4 as the starting
  centroids. You should compute the aggregate dissimilarity for each
  iteration. \\
\textbf{ANSWER:} \\
With the initial centers of 1 and 4: \\
$\Rightarrow$: Clusters: \{1,2\} And \{3,4,6,7,8,9\}; meanC1 = 1.5,
meanC2 = 4.83. Aggregate dissimilarity =
$0^2+1^2+1^2+0+2^2+3^2+4^2+5^2 = 56$. \\
$\Rightarrow$: Clusters: \{1,2,3\} And \{4,6,7,8,9\}; meanC1=2,
meanC2=6.8. Aggregate dissimilarity =
$2.75+0.689+1.369+4.71+27.43=36.95$. \\
$\Rightarrow$: Clusters: \{1,2,3,4\} And \{6,7,8,9\}; meanC1=2.5,
meanC2=7.5. Aggregate dissimilarity = 6+6.96=12.96. \\
$\Rightarrow$: Clusters: \{1,2,3,4\} and \{6,7,8,9\}. There is no change
compared with the last iteration, so we will end here. And the
Aggregate dissimilarity = 5+5 = 10. 

\item Apply the K-medoids algorithm for the following 1-dimensional
  points and k=2:1,2,3,4,6,7,8,9. Use 1 and 4 as the starting medoids.
  To make your job easier, assume that the algorithm is smart to
  choose the best replacement for each iteration. This will let you do
  the work using the smallest number of iterations. You should compute
  the aggregate dissimilarity for each iteration.\\ 
\textbf{ANSWER:} \\
With the initial centers of 1 and 4: \\
$\Rightarrow$: Clusters \{1,2\} and \{3,4,6,7,8,9\}; Aggregate
dissimilarity = 1+55= 56. \\
$\Rightarrow$ Replace 1 by 2: AggDis = 56. (good)\\
$\Rightarrow$ Replace 1 by 3: AggDis = 59. \\
$\Rightarrow$ Replace 1 by $6|7|8|9$: AggDis is large. \\
$\Rightarrow$ Replace 4 by $2|3$: AggDis is large. \\
$\Rightarrow$ Replace 4 by 6: AggDis = 20. Medoids: 2, 6.\\
$\Rightarrow$ Replace 4 by 7: AggDis = 12. Medoids: 2, 7.\\
$\Rightarrow$ Replace 4 by 8: AggDis = 12. Medoids: 2, 8.\\
$\Rightarrow$ Replace 2 by 3: AggDis = 12. Medoids: 2, 8.\\

So, the final medoids are 2, 6 with aggregate dissimilarity 12. 

\item Apply the bottom-up hierarchical algorithm for the following
  1-dimensional points and k=2:1,2,3,4,6,7,8,9.\\
  You should consider (a) using single linkage, and (b) complete
  linkage.\\ 
  For each of (a) and (b), draw the dendrogram. When there is a tie,
  choose clusters with "smaller means" before choosing clusters with
  larger means. \\
\textbf{ANSWER:} \\
Single linkage uses nearest neighbor technique, this method define
distance between clusters as the distance between the closest pair of
objects, where only pairs consisting of one object from each group is
considered. The distance $D(r,s)$ can be computed as: \\
$D(r,s) = Min{d(i,j) : Where object i is in cluster r and object j is
 cluster s}$ \\
At each stage of hierarchical clustering, the clusters r and s , for
which D(r,s) is minimum, are merged.

The complete linkage, also called farthest neighbor, clustering method
is the opposite of single linkage.  Distance between groups is now
defined as the distance between the most distant pair of objects, one
from each group. 
In the complete linkage method, D(r,s) is computed as \\
$D(r,s) = Max {d(i,j) : Where object i is in cluster r and object j is
 cluster s}$ \\
At each stage of hierarchical clustering, the clusters r and s , for
which D(r,s) is minimum, are merged.


\end{enumerate} % For whole homework. 

Marks available: 10 marks for each question. 
\end{document}

%%%%%%%%%%%%%%%%%%%%%%%%%%%%%%%%%%%%%%%%%%%%%%%%%%%%%%%%%%%%%
