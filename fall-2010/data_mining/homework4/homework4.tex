\documentclass{article}
% Change "article" to "report" to get rid of page number on title page
\usepackage{amsmath,amsfonts,amsthm,amssymb}
\usepackage{algorithmic,algorithm}
\usepackage{setspace}
\usepackage{Tabbing}
\usepackage{fancyhdr}
\usepackage{lastpage}
\usepackage{extramarks}
\usepackage{chngpage}
\usepackage{soul,color}
\usepackage{ulem}
\usepackage{graphicx,float,wrapfig}
\usepackage{amsfonts}
\usepackage{pifont}
\usepackage{booktabs}
\usepackage{hyperref}
\usepackage{pstricks,pst-node,pst-tree}

\newcommand{\tickYes}{\checkmark}
\newcommand{\tickNo}{\hspace{1pt}\ding{55}}

% In case you need to adjust margins:
\topmargin=-0.45in      %
\evensidemargin=0in     %
\oddsidemargin=0in      %
\textwidth=6.8in        %
\textheight=9.4in       %
\headsep=0.25in         %

% Homework Specific Information
\newcommand{\hmwkTitle}{Homework\ \#4}
\newcommand{\hmwkDueDate}{Nov.\ 10,\ 2010}
\newcommand{\hmwkClass}{Data Mining}
\newcommand{\hmwkClassTime}{MW\ 4:10-5:25pm}
\newcommand{\hmwkClassInstructor}{Guozhu\ Dong}
\newcommand{\hmwkAuthorName}{Shumin\ Guo}

% Setup the header and footer
\pagestyle{fancy}                                                       %
\lhead{\hmwkAuthorName}                                                 %
\chead{\hmwkClass\ - \hmwkTitle}  %
\rhead{Page\ \thepage\ of\ \pageref{LastPage}}                          %
\lfoot{\lastxmark}                                                      %
\cfoot{}                                                                %
\rfoot{}                          %
\renewcommand\headrulewidth{0.4pt}                                      %
%\renewcommand\footrulewidth{0.4pt}                                     %

%%%%%%%%%%%%%%%%%%%%%%%%%%%%%%%%%%%%%%%%%%%%%%%%%%%%%%%%%%%%%
% Make title
\title{\textbf{\hmwkClass:\ 
      \hmwkTitle}\\\normalsize\small{Due\ Date:\
    \hmwkDueDate}} 
\date{\today}
\author{\textbf{\hmwkAuthorName}}
%%%%%%%%%%%%%%%%%%%%%%%%%%%%%%%%%%%%%%%%%%%%%%%%%%%%%%%%%%%%%

\begin{document}
% \begin{spacing}{1.1}
\maketitle

\begin{enumerate}
\item (a) Define the clustering problem, and (b) define the measure
  commonly used to evaluate the quality of clustering results. \\
\textbf{ANSWER:}\\ 
Clustering is the process of grouping the data into classes or
clusters, so that objects within a cluster have high similarity in
comparison to one another but are very dissimilar to objects in other
clusters. Dissimilarities are assessed based on the attribute values
describing the objects. Often, distance measures are used.\\
The quality of clustering is estimated using a cost function that
measures the average dissimilarity between an object and the
representative object of its cluster, such as square error.

\item Compare the advantages and disadvantages of (a)K-means and (b)
  K-medoids for clustering. (c) Discuss a main challenge to both
  K-means and K-medoids algorithms.\\
\textbf{ANSWER:} \\
Advantage:\\
The k-means algorithm takes the input parameter, k, and partitions a
set of n objects into k clusters so that the resulting intracluster
similarity is high but the intercluster similarity is low. The
algorithm attempts to determine k partitions that minimize the
square-error function. It works well when the clusters are compact
clouds that are rather well separated from one another. The method is
relatively scalable and efficient in processing large data sets
because the computational complexity of the algorithm is $O(nkt)$, where
n is the total number of objects, k is the number of clusters, and t
is the number of iterations. Normally, k$\ll$n and t$\ll$n. The method often
terminates at a local optimum. 

Disadvantage:\\
The necessity for users to specify k,
the number of clusters in advance can be seen as a disadvantage. The
k-means method is not suitable for discovering clusters with
nonconvex shapes or clusters of very different size.  Moreover, it is
sensitive to noise and outlier data points because a small number of
such data can substantially influence the mean value.

In order to reduce the sensitivity to outliers, K-medoids methods,
instead of taking the mean value of the objects in a cluster as a
reference point, pick actual objects to represent the clusters, using
one representative object per cluster.

The k-medoids method is more robust than k-means in the presence of
noise and outliers, because a medoid is less influenced by outliers or
other extreme values than a mean. However, its processing is more
costly than the k-means method. Both methods require the user to
specify k, the number of clusters.


\item Compare the COBWEB algorithm against other clustering algorithms
  such as K-means and K-mediods. \\
\textbf{ANSWER:} \\


\item Briefly discuss how recent algorithms deal with clustering for
  large amount of data, or nonsphere shaped clusters.  \\
\textbf{ANSWER:} \\

\item Apply the K-means algorithm for the following 1-dimensional
  points and k=2:1,2,3,4,6,7,8,9. Use 1 and 4 as the starting
  centroids. You should compute the aggregate dissimilarity for each
  iteration. \\
\textbf{ANSWER:} \\

\item Apply the K-medoids algorithm for the following 1-dimensional
  points and k=2:1,2,3,4,6,7,8,9. Use 1 and 4 as the starting medoids.
  To make your job easier, assume that the algorithm is smart to
  choose the best replacement for each iteration. This will let you do
  the work using the smallest number of iterations. You should compute
  the aggregate dissimilarity for each iteration.\\ 
\textbf{ANSWER:} \\

\item Apply the bottom-up hierarchical algorithm for the following
  1-dimensional points and k=2:1,2,3,4,6,7,8,9.\\
  You should consider (a) using single linkage, and (b) complete
  linkage.\\ 
  For each of (a) and (b), draw the dendrogram. When there is a tie,
  choose clusters with "smaller means" before choosing cluters with
  larger means. \\
\textbf{ANSWER:} \\

\end{enumerate} % For whole homework. 

Marks available: 10 marks for each question. 
\end{document}

%%%%%%%%%%%%%%%%%%%%%%%%%%%%%%%%%%%%%%%%%%%%%%%%%%%%%%%%%%%%%
