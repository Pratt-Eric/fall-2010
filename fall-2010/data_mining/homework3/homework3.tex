\documentclass{article}
% Change "article" to "report" to get rid of page number on title page
\usepackage{amsmath,amsfonts,amsthm,amssymb}
\usepackage{algorithmic,algorithm}
\usepackage{setspace}
\usepackage{Tabbing}
\usepackage{fancyhdr}
\usepackage{lastpage}
\usepackage{extramarks}
\usepackage{chngpage}
\usepackage{soul,color}
\usepackage{ulem}
\usepackage{graphicx,float,wrapfig}
\usepackage{amsfonts}
\usepackage{pifont}
\usepackage{booktabs}
\usepackage{hyperref}

\newcommand{\tickYes}{\checkmark}
\newcommand{\tickNo}{\hspace{1pt}\ding{55}}

% In case you need to adjust margins:
\topmargin=-0.45in      %
\evensidemargin=0in     %
\oddsidemargin=0in      %
\textwidth=6.8in        %
\textheight=9.4in       %
\headsep=0.25in         %

% Homework Specific Information
\newcommand{\hmwkTitle}{Homework\ \#3}
\newcommand{\hmwkDueDate}{Oct.\ 27,\ 2010}
\newcommand{\hmwkClass}{Data Mining}
\newcommand{\hmwkClassTime}{MW\ 4:10-5:25pm}
\newcommand{\hmwkClassInstructor}{Guozhu\ Dong}
\newcommand{\hmwkAuthorName}{Shumin\ Guo}

% Setup the header and footer
\pagestyle{fancy}                                                       %
\lhead{\hmwkAuthorName}                                                 %
\chead{\hmwkClass\ - \hmwkTitle}  %
\rhead{Page\ \thepage\ of\ \pageref{LastPage}}                          %
\lfoot{\lastxmark}                                                      %
\cfoot{}                                                                %
\rfoot{}                          %
\renewcommand\headrulewidth{0.4pt}                                      %
%\renewcommand\footrulewidth{0.4pt}                                     %

%%%%%%%%%%%%%%%%%%%%%%%%%%%%%%%%%%%%%%%%%%%%%%%%%%%%%%%%%%%%%
% Make title
\title{\textbf{\hmwkClass:\ 
      \hmwkTitle}\\\normalsize\small{Due\ Date:\
    \hmwkDueDate}} 
\date{\today}
\author{\textbf{\hmwkAuthorName}}
%%%%%%%%%%%%%%%%%%%%%%%%%%%%%%%%%%%%%%%%%%%%%%%%%%%%%%%%%%%%%

\begin{document}
% \begin{spacing}{1.1}
\maketitle

\begin{enumerate}
\item Compare the advantages and disadvantages of eager classification
  (e.g., decision trees, bayesian, etc.) versus lazy classification
  (e.g., k-nearest neighbors). Eager systems build the model before
  they see the test data, lazy systems do not do model building until
  they see the test data. \\
\textbf{ANSWER:}


\item Why is tree pruning useful in decision tree induction? \\
\textbf{ANSWER:} 

\item Given a decision tree, you have the option of (a)converting the
  decision tree to rules and then pruning the resulting rules, or (b)
  pruning the decision tree and then converting the pruned tree to
  rules. Which method is better and why? 



\item Explain why is the NBC algorithm called naive? \\
\textbf{ANSWER:}

\item The following table consists of training data from an employee
  database.
\begin{table}[ht]
  \begin{center}
    \begin{tabular}{|c|c|c|c|}
      \hline {\it department} & {\it status}&{\it age}&{\it salary} \\
      \hline 
      sales & senior & 31-40 & Medium \\
      sales & junior & 21-30 & Low \\
      sales & junior & 31-40 & Low \\
      systems & junior & 21-30 & Medium \\
      systems & senior & 31-40 & High \\
      systems & junior & 21-30 & Medium \\
      systems & senior & 41-50 & High \\
      marketing & senior & 31-40 & Medium \\
      marketing & junior & 31-40 & Medium \\
      secretary & senior & 41-50 & Medium \\
      secretary & junior & 21-30 & Low \\
      \hline
    \end{tabular}
    \caption{Training Data of Employee Data.\label{tbl:dat_eple}}
  \end{center}
\end{table}

\begin{enumerate}
\item[(a)] Let {\it status} be the class attribute. Use the ID3
  algorithm to construct a decision tree from the given data.\\
\textbf{ANSWER:}

\item[(b)] Given an instance with the values systems, senior, and
  21-30 for the attribute {\it department},{\it status}, and
  {\it age}, respectively, what would be a naive bayesian
  classification for the {\it salary} of the sample be? You want to
  use "Laplace correction" for probability estimates. Show all the
  steps. \\
\textbf{ANSWER:}
\end{enumerate} % For question five. 

\item Given the data points (0,0/1), (1,0/1), (1,1/1), (4,3/2),
  (3,4/2), (1,4/2) where (x,y/c) gives a sample with x and y as
  attribute values and c the class label. Use k-NN to find the class
  label for (2,2/?) when k=1, and when k=5. Use Euclidean distance as
  the distance measure. \\ 
\textbf{ANSWER:}

\end{enumerate} % For whole homework. 

\end{document}

%%%%%%%%%%%%%%%%%%%%%%%%%%%%%%%%%%%%%%%%%%%%%%%%%%%%%%%%%%%%%
