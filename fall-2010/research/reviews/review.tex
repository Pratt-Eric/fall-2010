\documentclass[12pt]{article}
% Change "article" to "report" to get rid of page number on title page
\usepackage{amsmath,amsfonts,amsthm,amssymb}
\usepackage{setspace}
\usepackage{Tabbing}
\usepackage{fancyhdr}
\usepackage{lastpage}
\usepackage{extramarks}
\usepackage{chngpage}
\usepackage{soul,color}
\usepackage{titlesec}
\usepackage{graphicx,float,wrapfig}
\usepackage{url}

% In case you need to adjust margins:
\topmargin=-0.45in      %
\evensidemargin=-1in     %
\oddsidemargin=0in      %
\textwidth=7in        %
\textheight=9.0in       %
\headsep=0.25in         %

% Start of document. 
\begin{document}
\begin{center}                  % Header of file. 
\textbf{\Large{Privacy Protection of Social Networks - A Survey}} \\ 
\small\textsc{Shumin Guo} \\
\today
\end{center}

\section{Introduction}
With the advent of Web 2.0, social networks are becoming more and more
popular, it provides people with a virtual world to keep in touch with
friends, colleagues and relatives etc. And it is attracting more and
more users changing the way people communicate, work and
play. Facebook \cite{poke-facebook}, for example, has attracted more
than 600 million users \cite{facebooksite} from its publish in
2004. And as it grows, users both from the US and all around the
world are joining this online community. Still there are a lot of
other social networking sites, such as MySpace \cite{myspacesite},
which concentrates on music and entertainment, LinkedIn which focuses
on working professionals and Twitter \cite{twittersite}, a networking
service that let members send out short, 140-character messages called
tweets. These social networking sites gains great popularity in this
social network era.

The explosive growth of social networks brings new ways for people to
communicate with each other, social network users can publish their
personal information to utilize the services provided by social
networks, which have made people's personal relationships more visible
and quantifiable than ever before.  While on the other hand, they are
risking leakage of personal and sensitive information. In fact,
privacy incidentss has been reported over facebook
\cite{social-networking-report-economist}. Privacy is becoming a 
big concern when people are choosing to join in social networks.

This review article tries to figure out various privacy threats of
online social networks and methods to deal with these threats. Section
\ref{sec:threats} introduces privacy threats of online social
networks. Section \ref{sec:crypto} discusses cryptographic methods for
privacy preserving social networks. Section \ref{sec:accon}
introduces access control based based methods to control privacy of
social networks. Secion \ref{sec:platform} disscusses specific
privacy issues when social networks are used as an application
platform. And Section \ref{sec:summary} will wrap up this review and
point out possible research topics for online social network.

\section{Privacy threats of online social networks\label{sec:threats}}
\subsection{Leakage of Personal Private Information}
Users of online social networks want to share their private
information, either because they think that social network sites have
developed detailed sets of privacy controls that they can trust or
because they don't have sense of privacy protection on this online
communication channel. Some social network sites such as Facebook
provides options to users for them to personalize their private
settings eight private or access only by friend or by friend's friends
or totally public. Former research\cite{privacy-wizard} found that a lot of users
are not aware of the privacy access configuration and just leave it as
default, which will make their information public to everyone on the
social network sites. The traversal and random search/access feature
makes users' public information under privacy threat. 

Another fact is that although social network sites have privacy policy
related to users private information, they are not guaranteed to be
conducted due to reasons such as advertising. For example, in 2007
facebook breached users privacy information by tracking users purchase
activaties and send alert messages to their friends without
acknowledgement from user \cite{social-networking-report-economist}.

%Social networks can be used on businesses. And a lot of small
%businesses become bigger. 

The huge audience of social networks are natural resources for online
social network providers. For example, marketing via social networks
try to do targeted advertising to users by analyzing their character
from their online activities and profile information. Some of users
information is even sold to third parties for unpredicted use.  In
this sense, the social networking business model is paradoxic to the
privacy protection of social network users, which makes it hard for
private policies to be strictly conducted.

Besides of serving as communication platform, social network sites,
especially facebook and myspace are platforms for various social
application to run on such as multiplayer games. This kind of
scenarios brings possible leakage of private information, as social
identify is required to in order to use these social
applications. Users need to authorize applications to access their
sensitive information which might be abused by third party application
developers.

\subsection{Attacks on Social Network Users}
\textbf{Spamming and Phishing\cite{spamming-phishing-privacy}} \\
Spamming and phishing messages with malicious URL links or attachments
can be send to social network users. When users clicked the malicious
URL or download/execute the attachment, malicious programs or websites
can steal users personal information or do harm to users'
computer. But with the protection of security softwares and users'
awareness of the harm of suspected message, the success rate of this
blind spamming and phishing attack is usually low.

On the other hand, attackers are usually interested in obtaining the
sensitive personal information to do more targeted attack such as
phishing or spaming\cite{context-aware-spam}. 

\textbf{Identity Threft Attacks}\\
\cite{identity-theft-attack} discusses identity theft attacks. Friend
requests are send to users of social network in the hope to establish
friendship with the victim, then the attacker will be able to access
the sensitive information provided by victims.

Still, a more advanced, cross site attack can be initiated. Such
attacks use the feature that one person may have two identities and
thus friends on different social network sites. If he has registered
on the first social network while not on the second one, his identity
can be forged to make friend requests usually with large possibility
of success. Experiments shows that it is easy to do automatic crawling
and identity theft attack on the major social network sites.

\textbf{De-Anonymization Attacks}\\
In order to keep private of sensitive personal information, social
network users can control who can access their sensitive information
while keeping themselves anonimized within the social
network. \cite{group-deanonymize-attack} proposes an attack model which
uses group information to De-Anonymize users from social network. By
obtaining the group membership information of a particular user, it is
usually sufficient to uniquely identify the identity of this user.

In the group information based deanonymization attack, web browsing
history information is utilized to obtain group information of a
certain user. Experiments show that little effort is needed to do
De-Anonymization attacks, so it has the potential to affect millions
of social network users with group memberships.

Sometimes social links can become sensitive information when social
data are used for data mining
purposes. \cite{user-interaction-social-link} proposes the use of
interaction graph to impart meaning of social links by quantifying
user interactions, and showed that interaction graph can be used as a
better representation of user interactions of social network users.
Similarly, \cite{neighborhood-attack} dicusses neighborhood
attacks. The identity of a specific user can be identified by
obtaining some knowledge about the neighbors of a targeted victim and
the relationship among these neighbors.

\cite{anony-link-attack} proposes attack on the social link. Social
link attacks may happen when anonymized social network data are used
for data mining or other aggregational purposes. A lot of literature
including \cite{privacy-preserve-publishing-survey} have discussion on
how to protect sensitive information when publishing sensitive data.

\cite{privacy-sensitive-edge} discusses privacy issue of sensitive
edges of social networks and proposes methods based on randomization
and purtabation for privacy preservation. 

\textbf{Real Life Linkage} \\
Often times, social network activity can be linked to real life
scenario. Minimum information is need for such kind of real life
linkage attack if a lot of personal information is known about the
victim. \cite{real-life-social-network} compares online social network
and real life social network and points out the domain feature of the
real life social network and influences of online social network to
the real life social network. This paper also proposes that different
types of relationships of real life social network is a good
implication of the design of online social networks for privacy
purposes. It also proposes that causual relationships of both online
and real life social network is common and further confirms that the
interaction among social network users, which is also proposed in
\cite{user-interaction-social-link} are better implication of privacy
concerns for online social network users. 

\section{Cryptographic Privacy Protection \label{sec:crypto}}
Cryptogrpic means can be used to protect privacy of users. User
profiles and communication among friends are encrpted all the
time. \cite{noyb} proposes an approach to encrpt sensitive data and
share keys to authorize users who can access the original
information. In order to avoid tracking of cipher data, it also
proposes stegamogrphy to help avoid detection. Overhead might is
introduced for key and encrpytion dictionary management. 

\cite{reliable_email} introduces an authentication protocol for the
problem of spam mails. This protocol can also be used onto protection
of social network privacy. Authentication tokens are used in this
protocol. Outgoing mails will be attached an authentication
token. Mail reccipient will verify the the identity and the signature
of the token before accepting the mail.

It is clearly seen that cyptogrphic privacy protection methods can be
used as a method for social network privacy protection, but the
overhead is that key management can be overloaded when user has a lot
of friends and on the other hand the utility of online social network
are somehow compromised to private personal communication channel
which can be done by more secure instant messaging tools such as
Messenger. 

% From Access Control perspective. 
\section{Privacy Protection Through Access Control \label{sec:accon}}
While the privacy issue has raised a lot of concerns among OSN users,
current OSNs implement very basic access control models. Although easy
to use and understand, these models lack flexibility and often times
fail to protect the privacy of users. So, how to protect the privacy
of OSN users have become a very interesting and hot research topic.

Privacy protection of online social networks can be treated as access
control problem, but it is more than traditional access control used
by firewalls, in that it is more distributed and automated by user
themselves. And the fact\cite{facebook-privacy-settings} that social
network users are usually novice of information technology makes it
even harder for them to understand and configure their access control
settings. There are several research attempts to help user configure
their sensitive information by way of access control.

\cite{privacy-wizard} proposes a template for the design of social
networking privacy wizard. In this approach, a privacy-preference
model classifier is learned by doing active learning on the user input
data. By utilizing the power of machine learning, little user
interaction is need to configure their privacy preferences. Similarly,
\cite{social-spammer-machine-learning} uses machine learning to
identify spammers and malware disseminators. 

\subsection{Challenges for Access Control Privacy Protection}
Social Networks are built upon relations between users, so the access
control and privacy protection models should be built based on the
relationships among users. A lot of research proposals expressed the
access control requirements in terms of relationship paths existing in
the network and their depth. Also, some models support a notion of
trust/reputation as a further parameter for access control decisions.

Besides, the enforcement of relationship-based access control poses
interesting issues regarding privacy protection. A further issue is
the architecture on the support of access control. Unlike traditional
access control models such as rule based access control, the
centralized access control are no longer a suitable or efficient model
for OSNs. A decentralized privacy-aware access control mechanism
should be devised to enforce not only relationship-based access
control but also ensure the relationship privacy. 

According to a relationship-based access control model, access control
policies are specified in terms of relationships existing in the
OSN. Additionally, the depth of the relationship path is an important
parameter for some access control decisions, since users are usually
more inclined to share their resources with users not much away from
them. Still a further important parameter is represented by trust
among social network nodes. 

Moreover, in OSN, there exists relationship between users and
resources. So, a relationship-based access control model should
exploit not only the standard user to user relationships but should
also consider the various relationships and connection between users
and resources. 

As we have discussed earlier, a centralized access control model is
not suitable for OSNs, a decentralized model should be used, where
each user is responsible for policy specification and enforcement. But
the decentralized model need to verify the existence of specific paths
within an OSN. This kind of task may be very difficult and time
consuming in a decentralized manner. So, a further essential
requirement of access control enforcement is to devise efficient and
scalable implementation strategies.

On the other hand, the relationship-based access control also poses
threats to the relationships which are in general sensitive
information. As the relationship-based access control may require
disclosure of personal relationships. So, another requirement for
relationship-based access control model is to ensure that relationship
privacy is not breached during access control. Also, relationships may
have an associated trust value that must be protected during access
control. So, there will be a requirement to protect the relationship
trust level.

Also, the decentralized enforcement of access control should have some
mechanism to help a user to precisely estimate the other users's trust
level. And such mechanism should also preserve user privacy when
performing trust computation. 

\subsection{Some Review of Access Control Literature.}
In current available literatures, various access control models have
been proposed, such as access rule based model, client-based access
control model, ACL model, multi-level access control,
challenge-response-based model etc. 

Also, there are proposals on distance based access control, in which
requestors are divided into adjacent zones, and different zones have
different rules or policies for access control. 
A lot of literature also proposed cryptographic methods which are used
for relationship certificate exchange, attestation and so on. And there
are also a formalized access control model proposed in the literature,
in which the authers proposed several types of access control
policies, resources access control, search policy, tranversal policies
and communication policies. And recent research on semantic web also
provides opportunities for access control research.

But, it is important to know that most of the research related to
privacy in OSNs have focused on privacy-preserving techniques to mine
social network data, only a few provide solutions for privacy aware
access control. Among these privacy aware access control models, some
are policy based, while others consider trust protection. 

\section{Social Networks as Platform \label{sec:platform}}
Social networking sites can not only serve as a communication
platform, it can be used as a platform for third party applications. A
third party can use the published social API to develop their own
application. For example third parties can use the API of Facebook to
develop social games and other useful and interesting
applications. The social platform on the one hand brings vitality to
the social network, while on the other hand privacy threats are also
emerging. So privacy protection methods for this scenario are in
urgent requirement as more and more social application are developed
and more and more users joining the social cloud
platform. xBook\cite{xbook-social-platform} proposes methods to deal
with this issue. 

\section{Conclusion \label{sec:summary}}
In this survey, we discussed privacy threats of online social network
users, pointing out possible ways that information can be leaked to
malicious users and possible social attacks based on these sensitive
information. And we classify current attempts for online social
network privacy protection to two types: cyptographic and access
control, pointing out that the former methods brings overload to the
social network users and providers. The access control based access
control methods can utilize the power of machine learning techniques
to automate the configurations of privacy preferences. And initial
attempts show that this is better method for online social network
privacy protection. While on the other hand, access control of social
network is unlike the traditional access control systems. It is more
automatic, distributed and need more precise tunning to the content. 

Research in online social network privacy related area is still in its
infancy, there are a lot of issues to be explored. One is the design
of a satisfactory solution to relationship privacy protection during
path discovery. Another research direction is related to policy
administration. Due to the large size of social network and relations,
it is important to devise techniques and tools that can help user
evaluate the risk of unauthorized flows of information that the
specification of a policy or itsupdate may cause.

With the development of online social networks, privacy protection is
becoming one major concern. While on the other hand, social network
providers are usually reluctant to enforce strict privacy protection
policies because of business concerns, which undermines privacy
protection greatly. So, it might be advisable that social network
platforms are revised to allow third party as sensitive data
hoster. And on the other hand legal procedures can be used for the
enforcement of privacy protection for online social network users. 

\bibliographystyle{unsrt}
\bibliography{review}

\end{document}