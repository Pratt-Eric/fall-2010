\documentclass[12pt]{article}
% Change "article" to "report" to get rid of page number on title page
\usepackage{amsmath,amsfonts,amsthm,amssymb}
\usepackage{setspace}
\usepackage{Tabbing}
\usepackage{fancyhdr}
\usepackage{lastpage}
\usepackage{extramarks}
\usepackage{chngpage}
\usepackage{soul,color}
\usepackage{titlesec}
\usepackage{graphicx,float,wrapfig}
\usepackage{url}

% In case you need to adjust margins:
\topmargin=-0.45in      %
\evensidemargin=-1in     %
\oddsidemargin=0in      %
\textwidth=7in        %
\textheight=9.0in       %
\headsep=0.25in         %

% Start of document. 
\begin{document}
\begin{center}                  % Header of file. 
\textbf{Privacy Wizard for Social Networking Sites - Review} \\ 
\small\textsc{Shumin Guo} \\
\end{center}
% problem identification 
\section*{Problem Description}
Privacy of online social networks are becoming great concerns for
average people. And how to help them effectively define their privacy
settings is one of the urgent questions yet to be answered. This paper
is trying deal with this question by designing a privacy wizard which
by utilizing active learning method to incrementally define a model as
a classifier for privacy setting applications. 

\section*{Contribution of this paper}
This paper is one of the earliest attempts for privacy setting from
user's perspective. One of the main contribution is that it clearly
defines the challenges of privacy settings for average users. And an
automatic privacy setting wizard was designed. The wizard implements
the privacy-preference model by learning a classifier. It requires
very simple user interaction, and the classifier model improves with
more user input. And newly added friends can be adapted
gracefully. Another contribution of this paper is that it describes a
set of visualization and modification tools for advanced users which
can brings explicite meaning of privacy settings for these users.

The privacy wizard builds a model based on two kinds of information,
friends and user's profile settings. By letting user's tagging privacy
settings of part of his/her friends, the wizard learns a privacy
protection model. And by further using the structure of the social
graph, the wizard can achive a higher accuracy. 

% weakness description. 
\section*{Weaknesses of this paper}
This paper defines a wizard to actively learn a privacy setting
preference model. But it failed to consider contents, such as wall
posts which are dynamically generated. And although the social
structure are considered, it doesn't consider structures such as
friend of friend privacy settings which has embarked on facebook. 
 
\end{document} 