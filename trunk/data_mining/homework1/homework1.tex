\documentclass{article}
% Change "article" to "report" to get rid of page number on title page
\usepackage{amsmath,amsfonts,amsthm,amssymb}
\usepackage{algorithmic,algorithm}
\usepackage{setspace}
\usepackage{Tabbing}
\usepackage{fancyhdr}
\usepackage{lastpage}
\usepackage{extramarks}
\usepackage{chngpage}
\usepackage{soul,color}
\usepackage{ulem}
\usepackage{graphicx,float,wrapfig}
\usepackage{amsfonts}
\usepackage{pifont}

\newcommand{\tickYes}{\checkmark}
\newcommand{\tickNo}{\hspace{1pt}\ding{55}}

% In case you need to adjust margins:
\topmargin=-0.45in      %
\evensidemargin=0in     %
\oddsidemargin=0in      %
\textwidth=6.5in        %
\textheight=9.2in       %
\headsep=0.25in         %

% Homework Specific Information
\newcommand{\hmwkTitle}{Homework\ \#1}
\newcommand{\hmwkDueDate}{Sep.\ 22,\ 2010}
\newcommand{\hmwkClass}{Data Mining}
\newcommand{\hmwkClassTime}{MW\ 4:10-5:25pm}
\newcommand{\hmwkClassInstructor}{Guozhu\ Dong}
\newcommand{\hmwkAuthorName}{Shumin\ Guo}

% Setup the header and footer
\pagestyle{fancy}                                                       %
\lhead{\hmwkAuthorName}                                                 %
\chead{\hmwkClass\ - \hmwkTitle}  %
\rhead{Page\ \thepage\ of\ \pageref{LastPage}}                          %
\lfoot{\lastxmark}                                                      %
\cfoot{}                                                                %
\rfoot{}                          %
\renewcommand\headrulewidth{0.4pt}                                      %
%\renewcommand\footrulewidth{0.4pt}                                     %

% This is used to trace down (pin point) problems
% in latexing a document:
%\tracingall

%%%%%%%%%%%%%%%%%%%%%%%%%%%%%%%%%%%%%%%%%%%%%%%%%%%%%%%%%%%%%
% Some tools
\newcommand{\enterProblemHeader}[1]{\nobreak\extramarks{#1}{#1 continued on next page\ldots}\nobreak%
                                    \nobreak\extramarks{#1 (continued)}{#1 continued on next page\ldots}\nobreak}%
\newcommand{\exitProblemHeader}[1]{\nobreak\extramarks{#1 (continued)}{#1 continued on next page\ldots}\nobreak%
                                   \nobreak\extramarks{#1}{}\nobreak}%

\newlength{\labelLength}
\newcommand{\labelAnswer}[2]
  {\settowidth{\labelLength}{#1}%
   \addtolength{\labelLength}{0.25in}%
   \changetext{}{-\labelLength}{}{}{}%
   \noindent\fbox{\begin{minipage}[c]{\columnwidth}#2\end{minipage}}%
   \marginpar{\fbox{#1}}%

   % We put the blank space above in order to make sure this
   % \marginpar gets correctly placed.
   \changetext{}{+\labelLength}{}{}{}}%

\setcounter{secnumdepth}{0}
\newcommand{\homeworkProblemName}{}%
\newcounter{homeworkProblemCounter}%
\newenvironment{homeworkProblem}[1][]%
  {\stepcounter{homeworkProblemCounter}%
   \renewcommand{\homeworkProblemName}{#1}%
   \section{\homeworkProblemName}%
   \enterProblemHeader{\homeworkProblemName}}%
  {\exitProblemHeader{\homeworkProblemName}}%

\newcommand{\problemAnswer}[1]
  {\noindent\fbox{\begin{minipage}[c]{\columnwidth}#1\end{minipage}}}%

\newcommand{\problemLAnswer}[1]
  {\labelAnswer{\homeworkProblemName}{#1}}

\newcommand{\homeworkSectionName}{}%
\newlength{\homeworkSectionLabelLength}{}%
\newenvironment{homeworkSection}[1]%
  {% We put this space here to make sure we're not connected to the above.
   % Otherwise the changetext can do funny things to the other margin

   \renewcommand{\homeworkSectionName}{#1}%
   \settowidth{\homeworkSectionLabelLength}{\homeworkSectionName}%
   \addtolength{\homeworkSectionLabelLength}{0.25in}%
   \changetext{}{-\homeworkSectionLabelLength}{}{}{}%
   \subsection{\homeworkSectionName}%
   \enterProblemHeader{\homeworkProblemName\ [\homeworkSectionName]}}%
  {\enterProblemHeader{\homeworkProblemName}%

   % We put the blank space above in order to make sure this margin
   % change doesn't happen too soon (otherwise \sectionAnswer's can
   % get ugly about their \marginpar placement.
   \changetext{}{+\homeworkSectionLabelLength}{}{}{}}%

\newcommand{\sectionAnswer}[1]
  {% We put this space here to make sure we're disconnected from the previous
   % passage

   \noindent\fbox{\begin{minipage}[c]{\columnwidth}#1\end{minipage}}%
   \enterProblemHeader{\homeworkProblemName}\exitProblemHeader{\homeworkProblemName}%
   \marginpar{\fbox{\homeworkSectionName}}%

   % We put the blank space above in order to make sure this
   % \marginpar gets correctly placed.
   }%

%%%%%%%%%%%%%%%%%%%%%%%%%%%%%%%%%%%%%%%%%%%%%%%%%%%%%%%%%%%%%


%%%%%%%%%%%%%%%%%%%%%%%%%%%%%%%%%%%%%%%%%%%%%%%%%%%%%%%%%%%%%
% Make title
\title{\textbf{\hmwkClass:\ 
      \hmwkTitle}\\\normalsize\small{Due\ Date:\
    \hmwkDueDate}} 
\date{\today}
\author{\textbf{\hmwkAuthorName}}
%%%%%%%%%%%%%%%%%%%%%%%%%%%%%%%%%%%%%%%%%%%%%%%%%%%%%%%%%%%%%

\begin{document}
% \begin{spacing}{1.1}
\maketitle

% \clearpage
\section{Introduction - briefly answer the following questions}
\begin{enumerate}
\item What is Data Mining? What is the relation between data mining and
  KDD? \\

\textbf{ANSWER:} Generally, Data Mining is the process of analysing data
from different perspectives and summarizing it into useful
information. And technically, data mining is the process of finding
correlations and patterns among dozens of fields in large relational
database. Still another definitions is : a non-trivial process of
identifying valid, novel, useful and ultimately understandable patterns
in data. KDD (Knowledge Discovery from Database/Datasets) is a more
precise definition of data mining. 
\item Describe any three challenges of data mining. \\
% \renewcommand{\labelenumi}{\Roman{enumi}.}
% \renewcommand{\labelenumii}{\Roman{enumi}. \alph{enumii}}

\textbf{ANSWER:} 
\begin{itemize}
\item Increasing data dimensionality and data size. 
\item Various data forms. 
\item New data types. 
\end{itemize}

\item What is the need for data mining? \\

\textbf{ANSWER:} 
\begin{itemize}
\item More and more data are generated. 
\item Although a lot of data are generated, there is still a gap between
  stored data and knowledge, the transition won't occur automatically. 
\item Due to the volume and a lot of other challenges, manual data
  processing is almost impossible. 
\item The development of computer science and engineering generated
  high demands on the knowledge within the stored data. 
\item Fields such as finance, business etc. have high demands on seeking
  knowledge from the massive data. 
\end{itemize}
\end{enumerate}

\section{Data Pre-processing}
\begin{enumerate}
\item In real-world data, tuples (instances) with missing values for
  some attributes are a common occurrence. Brie°y describe various
  methods for handling this problem. \\
% \renewcommand{\labelitemi}{$\heartsuit$}
% \renewcommand{\labelitemii}{$\spadesuit$}

\textbf{ANSWER:} 
\begin{itemize}
\item Leave as is and treat missing value for each attribute A as a new
  value of A. 
\item Ignore/Remove the instances with missing value. 
\item Manual fix - assign a value with implicit meaning. 
\item Statistical methods to convert missing values - majority, most
  likely, mean nearest neighbor.
\end{itemize}

\item Suppose the data for analysis includes the attribute Age. The age
values for the data tuples (instances) are (in increasing order): 13,
15, 16, 16, 19, 20, 20, 21, 22, 22, 25, 25, 25, 25, 30, 33, 33, 35, 35,
35, 35, 36, 40, 45, 46, 52, 70. Use binning (by bin means) to smooth the
above data, using 5 bins and equi-density. (If the number of data tuples
is not a multiple of 5, evenly spread the extra tuples in the last few
bins.) Illustrate your steps, and comment on the effect of this technique
for the given data. \\

\textbf{ANSWER:} Equi-density binning expects to distribute data
samples into bins evenly, which means the number of data samples
should be the same for each bin, but for this example, we have 27
sample data to be distributed into 5 bins. So, we will need to assign 6
data samples for two bins and the other bins all have 5 samples. \\ 
As the data samples have been sorted increasingly, there is no need to
sort it. We get the following bin distribution table. \\ 
\begin{table}[h]
  \begin{center}
    \begin{tabular}{| c | c | c |}
      \hline ID\# & Samples & Bin \\
      \hline 1   & 13, 15, 16, 16, 19 & $(-\infty, 19)$ \\
      2 & 20, 20, 21, 22, 22 & $[19, 23)$ \\
      3 & 25, 25, 25, 25, 30 & $[23, 31)$ \\
      4 & 33, 33, 35, 35, 35, 35 & $[31, 35)$ \\ 
      5 & 36, 40, 45, 46, 52, 70 & $[35, +\infty)$ \\
      \hline
    \end{tabular}
    \caption{Data Distribution For Age Data with 5 Bins and
      Equi-Density Method.} 
  \end{center}
\end{table}

\textbf{\underline{COMMENTS:} } As in this example, age data is integer value, so
I simply picked the bin boundary to be the round down integer value of
the mean of the two sample values. E.g., for bin \#1 and \#2 the
boundary should be $\frac{19+20}{2} = 19.5$, I have rounded it to 19.

Data binning is used to smooth data, and seperate data into different
classes. So, the granularity of the data is increased, this can reduce
or smooth the noise contained within the data, but the panelty is that
it can also potentially affect the data that is of interest. 

\item Using the data for Age in Question 2, answer the following:  \\
\begin{itemize}
\item Use min-max normalization to transform the value 35 for age into the
  range [0.0; 1.0].\\

\textbf{ANSWER: } \\
For min-max normalization, we have formula: \\
$v' = \frac{(v - minA)\times (newMaxA - newMinA)}{(maxA - minA)} +
newMinA$ \\
So, for the above sample data, we have : \\ 
$v' = \frac{(35 - 13)\times (1.0 - 0.0)}{70 - 13} + 0.0 \approx 0.386$

\item Use z-score normalization to transform the value
  35 for age. The standard deviation of the ages is 12.94.\\

\textbf{ANSWER: } \\ 
For zero normalization, we have formula: \\ 
$v' = \frac{v - meanA}{stdevA}$, and the calculated mean value is
$meanA \approx 29.96$, so applying this formula, we have:\\
$v' = \frac{35 - 29.96}{12.94} \approx 0.39$.

\item Use normalization by decimal scaling to transform the value 35 for
  age. \\

\textbf{ANSWER: }\\
For decimal scaling, we have formula: \\
$v' = \frac{v}{10^k}$, of which k is the smallest number so that
$v'\in [-1,1]$. Applying this formula to this question, we can get: \\
$k = 2$, and so $v' = \frac{35}{10^2} = 0.35$. 

\item Comment on which method you would prefer to use for the given data,
giving reasons as to why. \\ 

\textbf{ANSWER: } Data normalization is one of the most important
steps in data preprocessing, especially when dealing with parameters
of different units and scales. Therefore, all parameters should have
the same scale for a fair comparison between them. Actually, all the
methods we used above have drawbacks, while scaling the normal data
into smaller intervals, those outliers will also be affected and
scaled into a scaler that is not even distinguishable to the worst
case. \\
Also, different normalization methods are suitable for different data
distribution properties. So, we need to justify before-hand what kind
of distribution our sample data conforms to, and then choose the best
normalization method for this kind of distribution. \\
Intuitively, for this example the age data will conform to Gaussian
distribution, which makes zero-score normalization the most suitable
method. 
\end{itemize}

\item We have the following (attribute-value,class) pairs [(0,P), (4,P),
(12,P), (14,N), (16,N), (16,N), (18,P), (24,N), (28,N)]. Consider two
possible splits $v_1 = 13$ and $v_2 = 15$ for discretizing the above data
attribute. Use entropy-based binning by binarization to find the
information gain for each split and decide which split is better. [In
real life, we need to consider all potential splits.]

\textbf{ANSWER:} \\
\uuline{For split $v_1 = 13$:} \\
$v_1$ devides the whole data set into two sets: \\
$S_1$ when $value \leq v_1$ and $S_2$ when $value > v_1$, and
specifically, we have: \\ 
$S_1 = [(0,P), (4,P), (12,P)]$ and $S_2 = [(14,N), (16,N), (16,N),
(18,P), (24,N), (28,N)]$ \\
And the information for this split is: \\
$IS(S_1,S_2) = \frac{|S_1|}{|S|}Entropy(S_1) +
\frac{|S_2|}{|S|}Entropy(S_2)$ \\
$= \frac{3}{9}\times 0 + \frac{6}{9}\times (-\frac{1}{6}log_2\frac{1}{6}
- \frac{5}{6}log_2\frac{5}{6})$ \\
$= 0 + \frac{2}{3}\times (0.4309 + 0.2191)$ \\
$= 0.43$ \\ 
The information gain of the split is: \\ 
Gain(v,S) = Entropy(S) - $IS(S_1,S_2)$ \\
$= -\frac{4}{9}log_2\frac{4}{9} - \frac{5}{9}log_2\frac{5}{9} - 0.43$ \\
$= 0.52 + 0.47 - 0.43$\\
$= 0.56$ \\

\uuline{For split $v_1 = 15$:} \\
$v_2$ devides the whole data set into two sets: \\
$S_1$ when $value \leq v_2$ and $S_2$ when $value > v_2$, and
specifically, we have: \\ 
$S_1 = [(0,P), (4,P), (12,P), (14,N)]$ and $S_2 = [(16,N), (16,N),
(18,P), (24,N), (28,N)]$ \\
And the information for this split is: \\
$IS(S_1,S_2) = \frac{|S_1|}{|S|}Entropy(S_1) +
\frac{|S_2|}{|S|}Entropy(S_2)$ \\
$= \frac{4}{9}\times (-\frac{3}{4}log_2\frac{3}{4} - \frac{1}{4}log_2\frac{1}{4}) +
\frac{5}{9}\times (-\frac{1}{5}log_2\frac{1}{5} - \frac{4}{5}log_2\frac{4}{5})$ \\
$= \frac{4}{9}\times (0.3113 + 0.5) + \frac{5}{9}\times (0.4644 + 0.2575)$ \\
$= 0.3606 + 0.401$ \\ 
$= 0.7616$

The information gain of the split is: \\ 
Gain(v,S) = Entropy(S) - $IS(S_1,S_2)$ \\ 
$= -\frac{4}{9}log_2\frac{4}{9} - \frac{5}{9}log_2\frac{5}{9} -
0.7616$ \\ 
$= 0.52 + 0.47 - 0.7616$ \\
$= 0.2284$

The split $v_1$ has higher information gain, so it is a better
choice. 

\item Consider the following data in the attribute-instance format. We
need to perform feature selection using Sequential Forward Selection
(SFS) with consistency (cRate) as the subset evaluation metric. What
will be the selected feature subset at the end of the second iteration?
\begin{table}[ht]
  \begin{center}
    \begin{tabular}{| c | c | c || c |}
      \hline F1 & F2 & F3 & C \\
      \hline 1 & 1 & 1 & 1 \\
      \hline 1 & 1 & 1 & 1 \\
      \hline 1 & 0 & 1 & 1 \\
      \hline 0 & 1 & 0 & 0 \\
      \hline 1 & 0 & 1 & 0 \\
      \hline 0 & 0 & 1 & 1 \\
      \hline 0 & 0 & 0 & 1 \\
      \hline 1 & 1 & 0 & 0 \\
      \hline 
    \end{tabular}
    \caption{Data Table for SFS}
  \end{center}
\end{table}

\textbf{ANSWER:} According to the definition of SFS, we need to
sequentially iterate all the features of this data set, in the first
iteration we choose feature F1, and the second iteration we will
choose F2 as the feature. We can get the following consistency result:

\begin{table}[ht]
  \begin{center}
    \begin{tabular}{| c | c | c || c | c | c || c | c | c |}
      \hline F1 & C & Consist & F2 & C & Consist & F3 & C & Consist \\
      \hline 1 & 1 & \tickYes & 1 & 1 & \tickYes & 1 & 1 & \tickYes \\
      \hline 1 & 1 & \tickYes & 1 & 1 & \tickYes & 1 & 1 & \tickYes \\
      \hline 1 & 1 & \tickYes & 0 & 1 & \tickYes & 1 & 1 & \tickYes \\
      \hline 0 & 0 & \tickNo  & 1 & 0 & \tickNo  & 0 & 0 & \tickYes \\
      \hline 1 & 0 & \tickNo  & 0 & 0 & \tickNo  & 1 & 0 & \tickNo  \\
      \hline 0 & 1 & \tickYes & 0 & 1 & \tickYes & 1 & 1 & \tickYes \\
      \hline 0 & 1 & \tickYes & 0 & 1 & \tickYes & 0 & 1 & \tickNo  \\
      \hline 1 & 0 & \tickNo  & 1 & 0 & \tickNo  & 0 & 0 & \tickYes \\
      \hline 
    \end{tabular}
    \caption{Consistency table for iterator one}
  \end{center}
\end{table}
So, the cRate after first iteration is $cRate_{F1}=\frac{5}{8}$
$cRate_{F2}=\frac{5}{8}$ and $cRate_{F3}=\frac{3}{4}$. So in this iteration
we will choose F3 as the starting point for next iteration. 

\begin{table}[ht]
  \begin{center}
    \begin{tabular}{| c | c | c || c | c | c || c | c | c |}
      \hline F1 & F3 & C & Consist & F2 & F3 & C & Consist \\
      \hline 1 & 1 & 1 & \tickYes & 1 & 1 & 1 & \tickYes \\
      \hline 1 & 1 & 1 & \tickYes & 1 & 1 & 1 & \tickYes \\
      \hline 1 & 1 & 1 & \tickYes & 0 & 1 & 1 & \tickYes \\
      \hline 0 & 0 & 0 & \tickNo  & 1 & 0 & 0 & \tickYes \\
      \hline 1 & 1 & 0 & \tickNo  & 0 & 1 & 0 & \tickNo  \\
      \hline 0 & 1 & 1 & \tickYes & 0 & 1 & 1 & \tickYes \\
      \hline 0 & 0 & 1 & \tickYes & 0 & 0 & 1 & \tickYes \\
      \hline 1 & 0 & 0 & \tickYes & 1 & 0 & 0 & \tickYes \\
      \hline 
    \end{tabular}
    \caption{Consistency table of feature two}
  \end{center}
\end{table}
So, the cRate after the second iteration is $cRate_{F1,F3}=\frac{3}{4}$
$cRate_{F2,F3}=\frac{7}{8}$. 

$\because$~ cRate$_{F1,F3} < $ cRate$_{F2,F3}$ \\
$\therefore$~ the selected feature after the second iteration will be
$\{F2,F3\}$. 
\end{enumerate}
\end{document}

%%%%%%%%%%%%%%%%%%%%%%%%%%%%%%%%%%%%%%%%%%%%%%%%%%%%%%%%%%%%%
