\documentclass{beamer}

% packages. 
% \usepackage[usenames]{color}
% \usepackage[utf8]{inputenc}
\usepackage{amsmath}
\usepackage{textcomp}
%\usepackage[utf8]{fontenc}
\usepackage{amssymb}
\usepackage{color}

% colors 
\definecolor{Blue}{rgb}{0,0,1}
\definecolor{Green}{rgb}{0.3,1,0.5}
\definecolor{Red}{rgb}{1,0,0}

\fontfamily{ptm}\selectfont
\setbeamerfont{frametitle}{family=\rmfamily,series=\bfseries,size={\fontsize{18}{18}}}
% to produce handouts. 
% \usepackage{pgfpages}
% \pgfpagesuselayout{4 on 1}[a4paper,landscape,border shrink=0.5cm]

% beamer configuration. 
%%% Presentation themes
%% Without navigation bars
% \usetheme{Boadilla}			% full page, bottom bar, light blue			
% \usetheme{Pittsburg}			% full page, default colors
%% Tree like navigation    	
% \usetheme{JuanLesPins}		% top tree, dark blue
%% Table of contents side bar
\usetheme{Hannover}			% left bar, light blue
% \usetheme{Goaettingen}		% right bar, light blue
%% Mini frame navigation   	
% \usetheme{Dresden}				% top frame, sober text, dark blue
% \usetheme{Singapore} 			% top frame with gradient, light blue
%% Section and subsection tables
% \usetheme{Warsaw} 				% circles, gradients, dark blue
% \usetheme{Luebeck} 			% rectangles, no gradient, dark blue

%%% Inner themes (lists, environments)
% \useinnertheme{circles}		% list elements in circles  
\useinnertheme{rectangles}	% list elements in rectangles  
% \useinnertheme{rounded}		% list elements in 3D spheres  
% \useinnertheme{inmargin}		% list elements and titles in left margin 

%%% Outer themes (head- footline, sidebars, frame title)
% \useoutertheme{infolines}	% lines at top and bottom, 3 parts each
% \useoutertheme{split}			% lines at top and bottom, 2
% parts each
% \useoutertheme{miniframes}	% navigation with points
% \useoutertheme{tree}			% navigation with a tree structure
\useoutertheme{sidebar}		% navigation in sidebar
% \useoutertheme{smoothbars}	% gradients on bars
% \useoutertheme{shadow}		% gradients under bars

%%% Colors themes 
%% Complete
% \usecolortheme{beetle}		% dark blue, grey background
% \usecolortheme{seagull}		% light grey
%% Inner                		
\usecolortheme{rose}			% light blue
% \usecolortheme{orchid}		% dark blue
%%	Outer                		
% \usecolortheme{seahorse}		% light blue
% \usecolortheme{whale}			% dark blue

%%% Dynamic display (of lists, environments...)
%% Default transition between successive items
% \beamerdefaultoverlayspecification{<+->}
%% How uncovered items are displayed
\setbeamercovered{invisible}	% not displayed
\setbeamercovered{transparent}	% transparent
% \setbeamercovered{dynamic}		% increased transparency away
% from the current item

% SHORTCUTS
% --------------------------------------------------------------------
\newcommand\defi[2]{\noindent\emph{\underline{Def}.} \texttt{#1}: #2}
\setcounter{tocdepth}{3}

\begin{document}

%% Document metadata
\title[]{\Huge{CEG 220 - C programming for Engineers}}
\subtitle{\scriptsize{Lab 1}}
\author[]{\fontfamily{ptm}\selectfont{Shumin Guo}\normalfont}
%\institute{Wright State University}
\date[\today]{\today}

%% Title frame
\frame[plain]{\titlepage}

%% Table of contents
% \begin{frame}[c]\frametitle{Outline}
%   \tableofcontents
% \end{frame}

\begin{frame}
\section{Objective} 

\end{frame}

/****************************************************************************
Author: your name         
Project: 1         
Class: CEG XXX
      File Name: rsmith-Proj-N.c        
Instructor: DeJongh                     
Due Date: [due date of current assignment]         
         
Overview:  A short statement of what the program does.  e.g.  This program accepts a series 
of numbers from standard input, displays them, calculates the average and standard deviation, 
and displays the result.


Program Structure:  A paragraph or two that describes how the program
works. 
\note{It, essentially, should provide a short walk-through of the
code.  e.g. The user is prompted for inputs; the data is read from the
keyboard using a while loop which checks for valid input; a series of
if statements (or perhaps a switch statement) are used to perform
tasks selected by the user; a set of functions is used to perform the
major tasks required; results are sent to the screen, formatted as a
table; the user is asked whether to write the data to a text file as
well.  etc, etc.}

The goal of this section is to give a reader a top level road map through your program.        
         
Major Variables:
        
Var1 represents xxxxx         
       Var2 represents xxxxx
etc.
Results/Observations (when required)
I observed …..etc…….

functions:
 /*******************************************************************
This function calculates the area of a circle, given the radius of the circle.
          It does this by multiplying the radius squared by pi.
Inputs:       radius of the circle
Outputs:    area of the circle with the given radius
      ********************************************************************/
double circleArea(double radius)

Insert comments to explain the algorithm used in major subsections, or any part, of a 
program where the logic of what you are doing is not obvious.  
\note{The goal is to allow someone 
(including you) to read your code and understand what the code is doing.
Precede single-line comments with //  
Use /*         .............               */ for multi-line comments (but do not use multi-line 
comments where they will distract from the code).}

Use meaningful identifier names (variables, function names, etc.). This means using names that 
describe the purpose of what they represent. \note{Identifiers should begin with a lowercase letter, and 
the subsequent “words” in the identifier name should begin with uppercase (such as lastName or 
calcSquareFootage).  Variables should be commented to explain their
use (eg applicable units).}

Coding Guidelines:
Be sure that variables are initialized before the program attempts to use their value.  This can be 
done when the variable is declared or with an assignment statement before the variable is first 
used.
Use enough parentheses in arithmetic expressions to make the order of operations clear.
Always include a prompt line in a program whenever the user is expected to enter data from the 
keyboard, and always echo the user’s input.
Names of symbolic constants and variables declared as constants should be all UPPERCASE.
Never hardcode numerical values; always use variables or symbolic
constants.

Code Readability:
All code within a block should be indented (generally 3 to 6 spaces, but be consisten); for 
example, indent all code within functions (including main), within if statements, and within 
loops.  This often means several levels of indenting within a block of code.  Follow the pattern 
used in your textbook.
Always place braces {} on a separate line, indented an equal amount to the previous line of code. 
Further indent all lines of code between brace pairs.  See your text for examples.


\begin{frame}{Policy}
\begin{enumerate}
\item Projects are due at the time and date specified on Pilot. 
\end{enumerate}

Although lab exercises are 
“officially due” Friday evening, your goal should be to turn them in
by the end of your lab section each week. 

If you do, you will earn 5 extra credit points for that lab, as long
as you earn at least 60\% on the material itself.
  
Your lab instructor will explain these procedures in lab during the
first week. 

Projects:  Projects are due on Saturday evenings by 11:55 pm. 

Late projects will be accepted up to 24 hours after the due time/date with a 
20\% grade penalty. No makeup exams unless there is a verifiable emergency. Exceptions to the late policy may 
be made only under the most unusual circumstances. All work must be your own; sharing of program code will 
result in a grade of "zero" for all involved. However, sharing ideas and general computer skills with others 
outside of class is encouraged. Students are expected to read and follow the Academic Integrity Policy: 
http://www.wright.edu/students/judicial/integrity.html

PROJECTS:
Project assignments will be posted on the Pilot Content Tab, usually at 8:00am on Sunday of the 
assignment week.  
All projects must be submitted through the Pilot drop box for the particular project, by the 
due date/time shown in Pilot.  Projects will normally be due Saturday, 11:55pm, of the due week 
as shown in Pilot.
To complete a submission on Pilot you must upload your files, and then submit them.  Failure
to click the submit button means they have not been submitted.  You should always verify that 
your files have been submitted correctly by checking your submission history.
Projects that are not submitted on time may be submitted up to 24 hours late, but they will incur 
a 20\% penalty.  IF you need to submit a project after its due date, but within the 24 late period, 
you must submit project to the Late Box.  This allows us to track late submissions.
If you feel that extenuating circumstances may justify a submittal outside these parameters, you 
should submit the file(s) to the Late Box, and email the course instructor immediately with your 
justification.  DO NOT EMAIL YOUR PROJECT FILES TO YOUR TA OR TO THE 
COURSE INSTRUCTOR.  

LABS:
Project assignments will be posted on the Pilot Content Tab, usually at 8:00am on Sunday of 
each week.  
All labs must be submitted through the Pilot drop box for the particular lab, by the due 
date/time shown in Pilot. To complete a submission on Pilot you must upload your files, and 
then submit them.  Failure to click the submit button means they have not been submitted. You 
should always verify that your files have been submitted correctly by checking your submission 
history.
Labs that are not submitted by the due date/time will receive no credit.
\end{frame}

Compilation (pick one)
Code Compiles
Serious Attempt but Doesn't Compile
Doesn't Compile
25%
+25
+15
+0
Execution (pick one)
Runs for all cases 
runs for some cases 
runs for no cases
15%
+15
+8
+0
Output (pick one)
Produces All Correct Output 
Produces mostly Correct Output 
Produces some correct output 
Produces no correct output
20%
+20
12
+5
+0
Programming (sum)
Logic and structure
Meaningful prompts for input
Use of functions when appropriate
use of control structures when appropriate
Well structured output
Error handling when appropriate
25%
+10
+3
+3
+3
+3
+3
Documentation
-------------------
(sum)
Complete header information
Meaningful identifiers and use of constants
Algorithm outline comments throughout
----------------------------------------------------
15%
+5
+5
+5
----

\begin{frame}
\section{Office Hours}
\end{frame}

\end{document}
