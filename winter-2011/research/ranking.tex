\documentclass[16pt]{report}
% Change "article" to "report" to get rid of page number on title page

% In case you need to adjust margins:
% \topmargin=-0.45in      %
% \evensidemargin=-1in     %
% \oddsidemargin=0in      %
% \textwidth=5in        %
% \textheight=9.0in       %
% \headsep=0.25in         %

\begin{document}
\textbf{\Large{Ranking Uncertain Data in Distributed Environments.}} \\ 

% Problem descriptions.  
\textbf{Problem description}

Collecting data from energy sensitive networks has been a challenging
problem. Previous research have showed that the data transmission power
constitutes most of the power consumed by network nodes. This paper is
trying to design an effective protocol to rank uncertain data in
energy sensitive distributed environment with the goal to reduce of
volume of data that is transmitted over the network nodes.

\textbf{Contributions}

% Contributions and methods proposed in this paper. 
After detailed analysis of queries, this paper identified two key
factors, global knowledge and decomposable computation, for effcient
in-network query processing. By utilizing these two factors, three
algorithms, global knowledge based (GKB), Partial Knowledge Based
(PKB) and hybrid for rank query processing are proposed over the tree
network topology environment. Global knowledge can be aggregated by
in-network computing, and partial knowledge is used to prune obsolete
objects, and hybrid method combines these two which thus can have a
better performance. 

\textbf{Defects of this paper} 

In this paper, extensive explanation about the proposed
methods/protocols are given. But the experiment methods and
configurations are not very properly explained. Also, this paper does
not consider the distribution of data which might also be a possible
factor for protocol performance. And also the topology should also be
specific as far as the results are concerned. 

There are several typos in this paper. For example, $r_7 = 7$ of the
$PW$ example, as there is no value 7 in the previous table. Another
typo is the formula $\sum_{j=1}^{t_i}p_{i,j} = 1$, it is not clear
what $t_i$ means here, should it be $|r_i|$, which is the number of
records. 

Also, several notations are missed in table one(Notations and Descriptions):
$t_i$ which appears several times in paper's algorithm description
part "let $V_i \leftarrow t_i$.

% Final rating score. 
\textbf{Final Rating}

Based on the major contributions and other technique factors, I give
this paper of score of 4 over 5. 

\end{document} 