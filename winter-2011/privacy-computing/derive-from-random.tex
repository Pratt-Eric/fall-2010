\documentclass[12pt]{article}
% Change "article" to "report" to get rid of page number on title page
\usepackage{amsmath,amsfonts,amsthm,amssymb}
\usepackage{setspace}
\usepackage{Tabbing}
\usepackage{fancyhdr}
\usepackage{lastpage}
\usepackage{extramarks}
\usepackage{chngpage}
\usepackage{soul,color}
\usepackage{titlesec}
\usepackage{graphicx,float,wrapfig}
\usepackage{url}

% In case you need to adjust margins:
\topmargin=-0.45in      %
\evensidemargin=-1in     %
\oddsidemargin=0in      %
\textwidth=7in        %
\textheight=9.0in       %
\headsep=0.25in         %

% Start of document. 
\begin{document}
\begin{center}                  % Header of file. 
\textbf{\large{Deriving Private Information from Randomized Data - Review}} \\ 
\small\textsc{Shumin Guo} \\
\end{center}

% Problems to deal with. 
Literature papers on privacy preserving data mining have proved
that randomization can be used to preserve privacy, but there are
critics that randomization methods may not be sufficient to
preserve privacy. So this paper is trying to study reasons that lead
to the weakness of privacy preservation of randomization methods, and
to understand how these features affect the privacy preservation 
property. It also discussed properties of data and randomization that
will have higher risk of disclosing privacy content even though they
are randomized.

% Methods proposed in this paper. 
In this paper, the author assert that a variety of information can
lead to the disclosure of private information including attribute
dependency, sample dependency, partial value disclosure and the
results of data mining operations. This paper focuses on the
correlations among attributes of data set. And it recommends three
corelation based methods to prove this proposal.  

Univariate data reconstruction scheme is a kind of algorithm which
does not utilize the correlations among data attributes. It is
proposed as a baseline method in this paper. As it only considers the
distribution of only one dimension, so if the attributes are highly
correlated, the risk of breaking privacy will be high. 

The second method is based on Principle Component Analysis (PCA),
which allows us to control the correlation among attributes, so that
we can study the relationships between correlation and privacy. 

Bayes estimate is proposed as a more general solution to this problem.
With Bayes estimate, original data is estimated from the randomized
data according to the rule that posterior probability be
maximized. Experimental results show that this method can achieve 
better performance than the PCA-based method. 

In order to study the properties of data and randomization that lead
to higher risk of privacy, a modified randomization scheme is proposed
to control the similarity between the original data and randomization
noise so that we can study the privacy preservation with different
similarity between these two. 

Experimental results show that higher correlation among data set
attributes can make privacy preservation less effective. Thus proved
the assert that correlation have effect on the privacy level. 

And results also show that the higher the similarity between data
and noise, the less accuracy for data reconstruction, and thus the
higher privacy preservation. Based on this experimental result, an
improved randomization scheme is proposed to enforce privacy
preservation. This scheme is trying to make the original data and the
noise have similar covariance matrix, and make them both concentrate
on the principal component of the original data so that higher privacy
level can be achieved. 

% Strength description of this paper. 
This paper is trying to analyze the factors that will make privacy
preservation less effective. Several factors are proposed and the
correlation among data set attributes are focused. In order to prove
and quantify the assertion of the efficacy of correlation among data
attributes, methematical proof are given. Extensive experimental
results show that the correlation of data attributes can cause the
leakage of private information. 

% Defect or untackled problems. 
The defect of this paper is that computation algorithms are not given
and the computational performance and complexity are not discussed. 

\end{document} 