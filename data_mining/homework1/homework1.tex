\documentclass{article}
% Change "article" to "report" to get rid of page number on title page
\usepackage{amsmath,amsfonts,amsthm,amssymb}
\usepackage{algorithmic,algorithm}
\usepackage{setspace}
\usepackage{Tabbing}
\usepackage{fancyhdr}
\usepackage{lastpage}
\usepackage{extramarks}
\usepackage{chngpage}
\usepackage{soul,color}
\usepackage{graphicx,float,wrapfig}

% In case you need to adjust margins:
\topmargin=-0.45in      %
\evensidemargin=0in     %
\oddsidemargin=0in      %
\textwidth=6.5in        %
\textheight=9.0in       %
\headsep=0.25in         %

% Homework Specific Information
\newcommand{\hmwkTitle}{Homework\ \#1}
\newcommand{\hmwkDueDate}{Sep.\ 22,\ 2010}
\newcommand{\hmwkClass}{Data Mining}
\newcommand{\hmwkClassTime}{MW\ 4:10-5:25pm}
\newcommand{\hmwkClassInstructor}{Guozhu\ Dong}
\newcommand{\hmwkAuthorName}{Shumin\ Guo}

% Setup the header and footer
\pagestyle{fancy}                                                       %
\lhead{\hmwkAuthorName}                                                 %
\chead{\hmwkClass\ - \hmwkTitle}  %
\rhead{Page\ \thepage\ of\ \pageref{LastPage}}                          %
\lfoot{\lastxmark}                                                      %
\cfoot{}                                                                %
\rfoot{}                          %
\renewcommand\headrulewidth{0.4pt}                                      %
%\renewcommand\footrulewidth{0.4pt}                                     %

% This is used to trace down (pin point) problems
% in latexing a document:
%\tracingall

%%%%%%%%%%%%%%%%%%%%%%%%%%%%%%%%%%%%%%%%%%%%%%%%%%%%%%%%%%%%%
% Some tools
\newcommand{\enterProblemHeader}[1]{\nobreak\extramarks{#1}{#1 continued on next page\ldots}\nobreak%
                                    \nobreak\extramarks{#1 (continued)}{#1 continued on next page\ldots}\nobreak}%
\newcommand{\exitProblemHeader}[1]{\nobreak\extramarks{#1 (continued)}{#1 continued on next page\ldots}\nobreak%
                                   \nobreak\extramarks{#1}{}\nobreak}%

\newlength{\labelLength}
\newcommand{\labelAnswer}[2]
  {\settowidth{\labelLength}{#1}%
   \addtolength{\labelLength}{0.25in}%
   \changetext{}{-\labelLength}{}{}{}%
   \noindent\fbox{\begin{minipage}[c]{\columnwidth}#2\end{minipage}}%
   \marginpar{\fbox{#1}}%

   % We put the blank space above in order to make sure this
   % \marginpar gets correctly placed.
   \changetext{}{+\labelLength}{}{}{}}%

\setcounter{secnumdepth}{0}
\newcommand{\homeworkProblemName}{}%
\newcounter{homeworkProblemCounter}%
\newenvironment{homeworkProblem}[1][]%
  {\stepcounter{homeworkProblemCounter}%
   \renewcommand{\homeworkProblemName}{#1}%
   \section{\homeworkProblemName}%
   \enterProblemHeader{\homeworkProblemName}}%
  {\exitProblemHeader{\homeworkProblemName}}%

\newcommand{\problemAnswer}[1]
  {\noindent\fbox{\begin{minipage}[c]{\columnwidth}#1\end{minipage}}}%

\newcommand{\problemLAnswer}[1]
  {\labelAnswer{\homeworkProblemName}{#1}}

\newcommand{\homeworkSectionName}{}%
\newlength{\homeworkSectionLabelLength}{}%
\newenvironment{homeworkSection}[1]%
  {% We put this space here to make sure we're not connected to the above.
   % Otherwise the changetext can do funny things to the other margin

   \renewcommand{\homeworkSectionName}{#1}%
   \settowidth{\homeworkSectionLabelLength}{\homeworkSectionName}%
   \addtolength{\homeworkSectionLabelLength}{0.25in}%
   \changetext{}{-\homeworkSectionLabelLength}{}{}{}%
   \subsection{\homeworkSectionName}%
   \enterProblemHeader{\homeworkProblemName\ [\homeworkSectionName]}}%
  {\enterProblemHeader{\homeworkProblemName}%

   % We put the blank space above in order to make sure this margin
   % change doesn't happen too soon (otherwise \sectionAnswer's can
   % get ugly about their \marginpar placement.
   \changetext{}{+\homeworkSectionLabelLength}{}{}{}}%

\newcommand{\sectionAnswer}[1]
  {% We put this space here to make sure we're disconnected from the previous
   % passage

   \noindent\fbox{\begin{minipage}[c]{\columnwidth}#1\end{minipage}}%
   \enterProblemHeader{\homeworkProblemName}\exitProblemHeader{\homeworkProblemName}%
   \marginpar{\fbox{\homeworkSectionName}}%

   % We put the blank space above in order to make sure this
   % \marginpar gets correctly placed.
   }%

%%%%%%%%%%%%%%%%%%%%%%%%%%%%%%%%%%%%%%%%%%%%%%%%%%%%%%%%%%%%%


%%%%%%%%%%%%%%%%%%%%%%%%%%%%%%%%%%%%%%%%%%%%%%%%%%%%%%%%%%%%%
% Make title
\title{\textmd{\textbf{\hmwkClass:\ 
      \hmwkTitle}}\\\normalsize\small{Due\ Data:\
    \hmwkDueDate}} 
\date{\today}
\author{\textbf{\hmwkAuthorName}}
%%%%%%%%%%%%%%%%%%%%%%%%%%%%%%%%%%%%%%%%%%%%%%%%%%%%%%%%%%%%%

\begin{document}
% \begin{spacing}{1.1}
\maketitle

% \clearpage
\section{Introduction - briefly answer the following questions}
\begin{enumerate}
\item What is Data Mining? What is the relation between data mining and
  KDD? 

\textbf{ANSWER:} Generally, Data Mining is the process of analysing data
from different perspectives and summarizing it into useful
information. And technically, data mining is the process of finding
correlations and patterns among dozens of fields in large relational
database. Still another definitions is : a non-trivial process of
identifying valid, novel, useful and ultimately understandable patterns
in data. KDD (Knowledge Discovery from Database/Datasets) is a more
precise definition of data mining. 
\item Describe any three challenges of data mining. 
% \renewcommand{\labelenumi}{\Roman{enumi}.}
% \renewcommand{\labelenumii}{\Roman{enumi}. \alph{enumii}}

\textbf{ANSWER:} 
\begin{itemize}
\item Increasing data dimensionality and data size. 
\item Various data forms. 
\item New data types. 
\end{itemize}

\item What is the need for data mining? 

\textbf{ANSWER:} 
\begin{itemize}
\item More and more data are generated. 
\item Although lot of data are generated, there is still a gap between
  stored data and knowledge, the transition won't occur automatically. 
\item Due to the volume and a lot of other challenges, manual data
  processing is almost impossible. 
\item The development of computer science and engineering generated
  high demands on the knowledge within the stored data. 
\item Fields such as finance, business etc. have high demands on seeking
  knowledge from the massive data. 
\end{itemize}

\end{enumerate}

\section{Data Pre-processing}
\begin{enumerate}
\item In real-world data, tuples (instances) with missing values for
  some attributes are a common occurrence. Brie°y describe various
  methods for handling this problem. 
% \renewcommand{\labelitemi}{$\heartsuit$}
% \renewcommand{\labelitemii}{$\spadesuit$}
\begin{itemize}
\item Leave as is and treat missing value for each attribute A as a new
  value of A. 
\item Ignore/Remove the instances with missing value. 
\item Manual fix - assign a value with implicit meaning. 
\item Statistical methods to convert missing values - majority, most
  likely, mean nearest neighbor.
\end{itemize}

\textbf{ANSWER:} 

\item Suppose the data for analysis includes the attribute Age. The age
values for the data tuples (instances) are (in increasing order): 13,
15, 16, 16, 19, 20, 20, 21, 22, 22, 25, 25, 25, 25, 30, 33, 33, 35, 35,
35, 35, 36, 40, 45, 46, 52, 70. Use binning (by bin means) to smooth the
above data, using 5 bins and equi-density. (If the number of data tuples
is not a multiple of 5, evenly spread the extra tuples in the last few
bins.) Illustrate your steps, and comment on the e®ect of this technique
for the given data. \\
\textbf{ANSWER:} 

\item Using the data for Age in Question 2, answer the following:  \\
\begin{itemize}
\item Use min-max normalization to transform the value 35 for age into the
  range [0:0; 1:0].  
\item Use z-score normalization to transform the value
  35 for age. The standard deviation of the ages is 12:94.  
\item Use normalization by decimal scaling to transform the value 35 for
  age.  
\item Comment on which method you would prefer to use for the given data,
giving reasons as to why.
\end{itemize}

\textbf{ANSWER:} 

\item We have the following (attribute-value,class) pairs [(0,P), (4,P),
(12,P), (14,N), (16,N), (16,N), (18,P), (24,N), (28,N)]. Consider two
possible splits v1 = 13 and v2 = 15 for discretizing the above data
attribute. Use entropy-based binning by binarization to ¯nd the
information gain for each split and decide which split is better. [In
real life, we need to consider all potential splits.]

\textbf{ANSWER:} 


\item Consider the following data in the attribute-instance format. We
need to perform feature selection using Sequential Forward Selection
(SFS) with consistency (cRate) as the subset evaluation metric. What
will be the selected feature subset at the end of the second iteration?
\begin{table}[h]
  \begin{center}
    \begin{tabular}{| c || c | c | c |}
      \hline F1 & F2 & F3 & C \\
      \hline 1 & 1 & 1 & 1 \\
      \hline 1 & 1 & 1 & 1 \\
      \hline 1 & 0 & 1 & 1 \\
      \hline 0 & 1 & 0 & 0 \\
      \hline 1 & 0 & 1 & 0 \\
      \hline 0 & 0 & 1 & 1 \\
      \hline 0 & 0 & 0 & 1 \\
      \hline 1 & 1 & 0 & 0 \\
      \hline 
    \end{tabular}
    \caption{Data Table for SFS}
  \end{center}
\end{table}
\textbf{ANSWER:} 
\end{enumerate}
\end{document}

%%%%%%%%%%%%%%%%%%%%%%%%%%%%%%%%%%%%%%%%%%%%%%%%%%%%%%%%%%%%%
