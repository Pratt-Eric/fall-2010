\documentclass{article}
\usepackage{amsmath,amsfonts,amsthm,amssymb}
\usepackage{algorithmic,algorithm}
\usepackage{setspace}
\usepackage{fancyhdr}
\usepackage{lastpage}
\usepackage{extramarks}
\usepackage{chngpage}
\usepackage{soul,color}
\usepackage{ulem}
\usepackage{graphicx,float,wrapfig}
\usepackage{pifont}
\usepackage{hyperref}
\usepackage{pstricks,pst-node,pst-tree}
\usepackage{pdftricks}
\usepackage{subfigure}
\usepackage{multicol}
\usepackage{enumerate}
\usepackage{pifont}
\usepackage{listings}
\usepackage{amsmath}
\usepackage{color}
%\usepackage{mathabx}
\usepackage{ulsy}
\usepackage{textcomp}
\newcommand{\tickYes}{\checkmark}
\newcommand{\tickNo}{\hspace{1pt}\ding{55}}

\definecolor{listinggray}{gray}{0.9}
\definecolor{lbcolor}{rgb}{0.9,0.9,0.9}

\lstset{
	%backgroundcolor=\color{lbcolor},
	tabsize=4,
	rulecolor=,
	language=c++,
        basicstyle=\setstretch{1},
        upquote=true,
        aboveskip={\baselineskip},
        columns=fixed,
        showstringspaces=false,
        extendedchars=true,
        breaklines=true,
        prebreak = \raisebox{0ex}[0ex][0ex]{\ensuremath{\hookleftarrow}},
        %frame=single,
        showtabs=false,
        showspaces=false,
        showstringspaces=false,
        identifierstyle=\ttfamily,
        keywordstyle=\color[rgb]{0,0,1},
        commentstyle=\color[rgb]{0.133,0.545,0.133},
        stringstyle=\color[rgb]{0.627,0.126,0.941},
}

% In case you need to adjust margins:
\topmargin=-0.45in      %
\evensidemargin=0in     %
\oddsidemargin=0in      %
\textwidth=7.0in        %
\textheight=9.2in       %
\headsep=0.25in         %

\renewcommand\headrulewidth{0.4pt}                                      %

%%%%%%%%%%%%%%%%%%%%%%%%%%%%%%%%%%%%%%%%%%%%%%%%%%%%%%%%%%%%%

\begin{document}
\begin{center}
\textbf{\Huge{CS 740 Execise Set 2}}\\
\textsc{Shumin Guo}
\end{center}

\begin{enumerate}
  \setcounter{enumi}{3}
\LARGE{\item Show: If $\lim\limits_{\substack{n \to
      \infty}}\frac{f(n)}{g(n)} = c$, with $0<c<\infty$, then $f\in
  \Theta(g)$ and $g\in \Theta(f)$.} 

\large{
\begin{align*}
  & \lim_{x \to \infty}\frac{f(n)}{g(n)} = c \\
  & \Rightarrow \lim_{x \to \infty}\frac{f(n)}{g(n)} = \frac{c.g(n)}{g(n)}  \\
  & \Rightarrow \lim_{x \to \infty}\frac{f(n)-c.g(n)}{g(n)} = 0 \\
  & \Rightarrow f(n)-c.g(n) \in O(g(n)),~[1.11(1)] \\
  & \Rightarrow \exists~ C~ and~ n_0~ s.t.~ f(n)-c.g(n)\le C.g(n)~for~all~
  n\ge n_0 \\
  & \Rightarrow f(n) \le (c+C).g(n)~ for~ all~ n\ge n_0
\end{align*}

Let $C^{\prime} = (c+C)$, we have $f(n) \le C^{\prime}.g(n)$ for all
$n\ge n_0$. 

$\therefore$ we have $f\in O(g)$. 

Similarly, 
\begin{align*}
  & \lim_{n \to \infty}\frac{f(n)}{g(n)} = c,~where~ c\in (0,\infty) \\
  & \Rightarrow \lim_{n \to \infty}\frac{g(n)}{f(n)} = 
  \frac{1}{c},[\frac{1}{c}\in (0, \infty)]. \\
  & \Rightarrow \lim_{n\to \infty}\frac{c.g(n)}{c.f(n)} =
  \frac{f(n)}{c.f(n)} \\
  & \Rightarrow \lim_{n\to \infty}\frac{c.g(n)-f(n)}{c.f(n)} = 0 \\
  & \Rightarrow c.g(n)-f(n)\in O(c.f(n))~ [1.11(1)]\\ 
  & \Rightarrow \exists C~and~n_0~ s.t.~ c.g(n)-f(n)\le
  C.cf(n)~ for~ all~ n\ge n_0. \\
  & \Rightarrow c.g(n) \le f(n) + C.c.f(n)~ for~ all~ n\ge n_0.  \\ 
  & \Rightarrow g(n) \le \frac{1}{c}(1+C.c).f(n)~ for~ all~
  n\ge n_0 \\
  & \Rightarrow Let~ C^{\prime} = \frac{1}{c}(1+C.c),~ we~
  have~ g(n) \le C^{\prime}f(n)~ for~ all~ n\ge n_0. \\
  & \Rightarrow g(n)\in O(f(n)).  \\
  & \Rightarrow f\in \Theta(g)~ and~ g\in \Theta(f)~ (1.5). 
\end{align*}

% According to the definition of $\Theta$-notation, we have $f\in
% \Theta(g)$ and $g\in \Theta(f)$.
}

% \LARGE{\item Show: If $\lim_{n \to \infty}\frac{f(n)}{g(n)}=\infty$,
%   then $f\notin O(g)$ and $g\in O(f)$.}

% \large{
% \[\lim_{n\to \infty}\frac{f(n)}{g(n)} = \infty
% \Rightarrow \lim_{n\to \infty}\frac{g(n)}{f(n)} = 0\] 
% According to Theorem 1.11(1), we have $g\in O(f)$. 

% Let's assume $f\in O(g)$, then $\exists C$ and $n_0$, s.t. $f(n)\le
% C.g(n)$ for all $n\ge n_0$. 

% $\Rightarrow \frac{f(n)}{g(n)}\le C$ for all $n\ge n_0$. 

% \[\therefore \lim_{n\to \infty}\frac{f(n)}{g(n)}\le C
% \Rightarrow \lim_{n\to \infty}\frac{g(n)}{f(n)}\le \frac{1}{C} \neq 0 \blitzb
% \]
% }

\end{enumerate}
\end{document}